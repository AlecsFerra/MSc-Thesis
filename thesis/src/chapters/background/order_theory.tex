\section{Order theory}\label{sec:backround:order_theory}

When defining the semantics of programming languages, the theory of 
\textit{partially ordered sets} and \textit{lattices} is fundamental. These 
concepts are at the core of denotational semantics \cite{Scott70} and 
\textit{Abstract Interpretation} \cite{Cousot77}, where the semantics of 
programming languages and abstract interpreters are defined as monotone 
functions over some complete lattice.

\subsection{Partial Orders}

\begin{definition}[Partial order]
  A partial order on a set $X$ is a relation $\leq \subseteq X \times X$ 
  such that the following properties hold:
  \begin{itemize}
    \item Reflexivity: $\forall x \in X, \; (x, x) \in \; \leq$
    \item Anti-symmetry: $\forall x, y \in X, \; (x, y) \in \; \leq \mand
      (y, x) \in \; \leq \implies x = y$
    \item Transitivity: $\forall x, y, z \in X, \; (x, y) \in \; \leq \mand 
      (y, z) \in \; \leq \implies (x, z) \in \;\leq$
  \end{itemize}
    
\end{definition}

Given a partial order $\leq$, we will use $\geq$ to denote the converse 
relation $\{ (y, x) \mid (x, y) \in \;\leq \}$ and $<$ to denote 
$\{ (x, y) \mid (x, y) \in \;\leq \; \text{and} \; x \neq y \}$.

From now on we will use the notation $x R y$ to indicate $(x, y) \in R$.

\begin{definition}[Partially ordered set]
  A partially ordered set (or poset) is a pair $(X, \leq)$ in which $\leq$ is a 
  partial order on $X$.
\end{definition}

\begin{definition}[Monotone function]
  Given two ordered sets $(X, \leq)$ and $(Y, \sqsubseteq)$, a function 
  $f : X \to Y$ is said to be monotone if $x \leq y \implies f(x) \sqsubseteq 
  f(y)$.
\end{definition}

\begin{definition}[Galois connection]
  Let $(C, \sqsubseteq)$ and $(A, \leq)$ be two partially ordered sets, a 
  Galois connection written $\langle C, \sqsubseteq \rangle 
  \galois{\alpha}{\gamma} \langle A, \leq \rangle$, are a pair of functions:
  $\gamma : A \to D$ and $\alpha : D \to A$ such that:
  \begin{itemize}
    \item $\gamma$ is monotone
    \item $\alpha$ is monotone
    \item $\forall c \in C$ $c \sqsubseteq \gamma(\alpha(c))$
    \item $\forall a \in A$ $a \leq \alpha(\gamma(a))$
  \end{itemize}
\end{definition}

\begin{definition}[Galois Insertion]
  Let $\langle C, \sqsubseteq \rangle \galois{\alpha}{\gamma} \langle A, \leq 
  \rangle$, be a Galois connection, a Galois insertion written 
  $\langle C, \sqsubseteq \rangle \galoiS{\alpha}{\gamma} \langle A, \leq \rangle$
  are a pair of functions: $\gamma : A \to D$ and $\alpha : D \to A$ such that:
  \begin{itemize}
    \item $(\gamma, \alpha)$ are a Galois connection
    \item $\alpha \circ \gamma = id$
  \end{itemize}
\end{definition}

\begin{definition}[Fixpoint]
  Given a function $f : X \to X$, a fixpoint of $f$ is an element $x \in X$ 
  such that $x = f(x)$.

  We denote the set of all fixpoints of a function as $\fix(f) = 
  \{ x \mid x \in X \mand x = f(x) \}$.
\end{definition}

\begin{definition}[Least and Greatest fixpoints]
  Given a function $f : X \to X$,
  \begin{itemize}
    \item We denote the \textit{least fixpoint} as $\lfp(f) = \min \fix(f)$.
    \item We denote the \textit{greatest fixpoint} as $\gfp(f) = \max \fix(f)$.
  \end{itemize}
\end{definition}

\subsection{Lattices}

\begin{definition}[Meet-semilattice]
  A poset $(X, \leq)$ is a meet-semilattice if $\forall x, y \in X, \exists z 
  \in X$ such that $z = \inf \{x, y\}$, called the \textit{meet}.

  Usually, the meet of two elements $x, y \in X$ is written as $x \meet y$.
\end{definition}

\begin{definition}[Join-semilattice]
  A poset $(X, \leq)$ is a join-semilattice if $\forall x, y \in X, \exists z 
  \in X$ such that $z = \sup \{x, y\}$, called the \textit{join} or 
  \textit{least upper bound}.

  Usually, the join of two elements $x, y \in X$ is written as $x \join y$.
\end{definition}

\begin{observation}
  Both join and meet operations are idempotent, associative, and commutative.
\end{observation}

\begin{definition}[Lattice]
  A poset $(X, \leq)$ is a lattice if it is both a join-semilattice and a 
  meet-semilattice.
\end{definition}

\begin{definition}[Complete lattice]
  A lattice $(X, \leq)$ is said to be complete if $\forall \; Y \subseteq X$:
  \begin{itemize}
    \item $\exists \; z \in X$ such that $z = \sup \; Y$
    \item $\exists \; z \in X$ such that $z = \inf \; Y$
  \end{itemize}

  We denote the \textit{least element} or \textit{bottom} as $\bot = \inf \; X$ 
  and the \textit{greatest element} or \textit{top} as $\top = \sup \; X$.
\end{definition}

\begin{observation}
  A complete lattice cant be empty.
\end{observation}

\begin{definition}[Point-wise lift]
  Given a complete lattice $L$ and a set $A$ we call \textit{point-wise} lift
  of $L$ the set of all functions $A \to L$ ordered point-wise $f \leq g \iff
  \forall a \in A \; f(a) \leq f(g)$.
\end{definition}

\begin{theorem}[Point-wise fixpoint]
  The leaft-fixpoint and greatest fixpoint on some point-wise lifted lattice on 
  a monotone function defined point-wise is the point-wise lift of the function.

  $$\lfp(\lambda p' a . f(p'(a))) = \lambda a . \lfp(\lambda p' . f(a))$$
  $$\gfp(\lambda p' a . f(p'(a))) = \lambda a . \gfp(\lambda p' . f(a))$$
\end{theorem}

\begin{theorem}[Knaster-Tarski theorem]
  \label{thm:knaster}
  Let $(L, \leq)$ be a complete lattice and let $f : L \to L$ be a monotone 
  function. Then $(\text{fix}(f), \leq)$ is also a complete lattice.
\end{theorem}

Two direct consequences that both the greatest and the least fixpoint of
$f$ exists and are respectively $\top$ and $\bot$ of $\fix(f)$.

\begin{theorem}[Post-fixpoint inequality]
  \label{thm:post-lfp}
  Let $f$ be a monotone function on a complete lattice then
  $$f(x) \leq x \implies \lfp(f) \leq x$$
\end{theorem}
\begin{proof}
  By theorem \ref{thm:knaster} $\lfp(f) = \bigwedge\{ y \mid y \geq f(y) \}$
  thus $\lfp(f) \leq x$ since $x \in \{ y \mid y \geq f(y) \}$.
\end{proof}

\begin{theorem}[$\lfp$ monotonicity]
  \label{thm:lfp-mono}
  Let $L$ be a complete lattice then if $P \leq Q$ and $f$ is monotone
  $$lfp(\lambda X. P \join f(X)) \leq lfp(\lambda X. Q \join f(X))$$
\end{theorem}
\begin{proof}
  \begin{align*}
    P \join f(\lfp(\lambda X . Q \join f(X)))
      &\leq Q \join f(\lfp(\lambda X . Q \join f(X)))
      &\text{Since $P \leq Q$} \\
      &= \lfp(\lambda X . Q \join f(X))
      &\text{By definition of fixpoint} \\
  \end{align*}
  Thus by theorem \ref{thm:post-lfp} pick $f = \lambda X . P \join f(X)$ and
  $x = lfp(\lambda X. Q \join f(X))$ it follows that
  $lfp(\lambda X. P \join f(X)) \leq lfp(\lambda X. Q \join f(X))$.
\end{proof}

