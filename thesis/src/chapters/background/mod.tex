\chapter*{Introduction}


The verification of program correctness is a critical task in computer science.
Ensuring that software behaves as expected under all possible conditions is
fundamental in a society that increasingly relies on computer programs.
Programmers often reason about the behavior of their programs at an intuitive
level. While this is definitely better than not reasoning at all, intuition
alone becomes insufficient as the size of programs grows.

Writing tests for programs is definitely a useful task, but at best, it can
show the presence of bugs, not prove their absence. We cannot feasibly write a
test for every possible input of the program. To offer a guarantee of the
absence of undesired behavior, we need sound logical models rooted in logic.
The field of formal methods in computer science aims to develop the logical
tools necessary to prove properties of software systems.

Hoare logic, first popularized by Hoare in the late 60s, provides a set of
logical rules to reason about the correctness of computer programs. Hoare logic
formalizes, with axioms and inference rules, the relationship between the
initial and final states after running a program.

Hoare logic, beyond being one of the first, is arguably also one of the most
influential ideas in the field of software verification. It created the whole
field of program logics—systems of logical rules aimed at proving properties of
programs. Over the years, modifications of Hoare logic have been developed,
sometimes to support new language features such as dynamic memory allocation
and pointers, or to prove different properties such as equivalence between
programs or properties of multiple executions. Every time Hoare logic is
modified, it is necessary to prove again that the proof system indeed proves
properties about the program (soundness) and that the proof system is powerful
enough to prove the properties of interest (completeness).

Most modifications of Hoare logic usually do not alter the fundamental proof
principles of the system. Instead, they often extend the assertion language to
express new properties and add new commands to support new features in
different programming languages.

We introduce Abstract Hoare Logic, which aims to be a framework general enough
to serve as an extensible platform for constructing new Hoare-like logics
without the burden of proving soundness and completeness anew. We demonstrate,
by example, how some properties that are not expressible in standard Hoare
logic can be simply instantiated within Abstract Hoare Logic, while keeping the
proof system as simple as possible.

The theory of Abstract Hoare Logic is deeply connected to the theory of
abstract interpretation. The semantics of the language is defined as an
inductive abstract interpreter, and the validity of the Abstract Hoare triples
depends on it. By not using the strongest postcondition directly, we are able
to reason about properties that are not expressible in the powerset of the
states, such as hyperproperties.


This thesis is subdivided as follows:
\begin{itemize}
  \item In Chapter 1, we introduce the basic mathematical background of order
    theory and abstract interpretation.

  \item In Chapter 2, we introduce standard Hoare logic and the general
    framework of Abstract Hoare Logic: the extensible $\lang$ language, its
    syntax and semantics, the generalization of the strongest postcondition,
    and finally, the actual Abstract Hoare Logic and its proof system, proving
    the general results of soundness and relative completeness.

  \item In Chapter 3, we show some interesting instantiations of Abstract Hoare
    Logic: we demonstrate that it is possible to obtain program logic where the
    implication is decidable, thus making the goal of checking a derivation
    computable; we show how to obtain a proof system for hyperproperties (and
    introduce the concept of the strongest hyper postcondition); and we show
    that it is possible to obtain a proof system for partial incorrectness.

  \item In Chapter 4, we show how to enrich the barebones proof system of
    Abstract Hoare Logic by adding more restrictions on the assertion language
    or the semantics.

  \item In Chapter 5, we show how to reuse the idea of Abstract Hoare Logic to
    generalize proof systems for backward reasoning.

  \item In Chapter 6, we provide a brief recap of the most important points of
    the thesis. We discuss possible extensions to the framework of Abstract
    Hoare Logic and, finally, we examine the relationship of Abstract Hoare
    Logic with other similar work.
\end{itemize}

\chapter{Background}

\subimport{./}{order_theory}
\subimport{./}{abstract_interpretation}
