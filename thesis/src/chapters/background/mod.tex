\chapter{Introduction and Background}

The verification of program correctness is a critical task in computer science, 
ensuring that software behaves as expected under all possible conditions. 
Hoare logic, introduced by Hoare in the late 1960s, has been a foundational tool 
in this area. It provides a formal system for reasoning about the correctness of 
programs by using assertions about the program state before and after execution.

However, traditional Hoare logic is limited to reasoning about properties that
are expressible as elements of $\pow{\states}$ which restricts its applicability 
in modern contexts where properties of interest often involve relationships 
between multiple executions or assertions must be computed automatically by some 
program.

To address these limitations, this thesis presents an extension of Hoare logic 
into an abstract framework that can handle a wider variety of program properties, 
including those involving multiple executions. By incorporating principles from 
order theory and abstract interpretation, we develop an abstract Hoare logic 
framework. This framework allows us to reason about program properties in a more 
general setting, providing a sound and relatively complete method for program 
verification.


We also introduce and explore the concept of hyperproperties within this 
framework. Hyperproperties, which describe relationships between different 
executions of a program, are crucial for expressing and verifying complex 
security and correctness properties. Our framework can be specialized to logic 
for hyperproperties, and in doing so, we also develop an inductive definition 
for the strongest hyper postcondition of a program, which, to the best of our 
knowledge, has not been previously achieved.

The thesis is structured as follows: We begin with a review of order theory and 
abstract interpretation, which form the theoretical foundation for our work. We 
then introduce the minimal version of the abstract Hoare logic framework and 
demonstrate its instantiation for various abstract domains. Finally, we provide 
some useful extensions to the proof system that make its usage easier.

\subimport{./}{order_theory}
\subimport{./}{abstract_interpretation}
