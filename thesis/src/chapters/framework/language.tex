\section{The $\lang$ programming language}

\subsection{Syntax}

The $\lang$ language is inspired by Dijkstra's guarded command languages
\cite{Dijkstra74} but with the goal of beeing as general as possible by beeing
parametric on a set of \textit{base commands}. The $\lang$ language is general
enough to describe any imperative non deterministic programming language.

\begin{definition}[$\lang$ language syntax]
  Given a set $Base$ of base commands, the set on valid $\lang$ programs is 
  defined by the following inductive definition:

  \begin{align*}
    \lang \; & \cceq \sskip        & \text{Skip}\\
             & \smid b             & \text{Base command}\\
             & \smid C_1 \fcmp C_2 & \text{Program composition}\\
             & \smid C_1 + C_2     & \text{Non deterministic choice}\\
             & \smid C^\fix        & \text{Iteration}\\
  \end{align*}

  Where $C, C_1, C_2 \in \lang$ and $b \in Base$.

  \begin{example} \label{exmp:base-commands-syntax}
    Usually the set of base commands contains a command $e ?$ to discard 
    execution that don't satisfy the predicate $e$ and $x \ass y$ to assing the 
    value $y$ to the variable $x$.
  \end{example}
\end{definition}


\subsection{Semantics}

Fixed a set $\states$ of states (usually a collection of associations between
variables names and values) and a family of partial functions $\bsem{\cdot} 
: \states \topartial \states$ we can define the denotational semantics of programs 
in $\lang$, the \textit{collecting semantics} is a function $\sem{\cdot} : \lang 
\to \pow{\states} \to \pow{\states}$ that associates a program $C$ and set of 
initial states to the set of states reached after executing the program $C$ 
from the initial states.

\begin{definition}[$\lang$ denotational semantics]
  \label{def:deno}
  Given a set $\states$ of states and a family of partial functions 
  $\bsem{b} : \states \topartial \states$ $\forall b \in Base$ the denotational 
  semantics is defined as follows:

  \begin{align*}
      \sem{\cdot}         & \;\;:\; \lang \to \pow{\states} \to \pow{\states} \\
      \sem{\sskip}        &\defeq id \\
      \sem{b}             &\defeq \lambda P . \{ \bsem{b}(p)\downarrow \; \mid 
        p \in P \} \\
      \sem{C_1 \fcmp C_2} &\defeq \sem{C_2} \circ \sem{C_1} \\
      \sem{C_1 + C_2}     &\defeq \lambda P . \sem{C_1} P \cup \sem{C_2} P \\
      \sem{C^\fix}        &\defeq \lambda P . \lfp(\lambda P'. P \cup \sem{C} P')
  \end{align*}

  \begin{example}
    We can define the semantics of the base commands introduced in 
    \ref{exmp:base-commands-syntax} as:
    $$\bsem{e ?}(\sigma) \defeq \begin{cases}
      \; \sigma & \sigma \models e \\
      \; \uparrow & otherwise
    \end{cases}$$
    $$\bsem{x \ass y}(\sigma) \defeq \sigma[x/eval(y, \sigma)]$$
    Where $eval$ is some evaluate function for the expressions on the left-hand
    side of assignments.
  \end{example}

  \begin{theorem}[Complete lattice]
    \label{thm:sem-lattice}
    $(\pow{\states}, \subseteq)$ is a complete lattice.
  \end{theorem}
  \begin{proof}
    To prove that $(\pow{\states}, \subseteq)$ is a complete lattice we exhibit:
    $\forall P \subseteq \pow{states}$
    \begin{itemize}
      \item $\inf P = \bigcap P$, it's clearly a lowerbound, and it's the 
        greatest since any other set $Z \supsetneq \bigcap P$ contains some
        not in any of the elements in $P$.
      \item $\sup P = \bigcup P$, it's clearly an upper bound, and it's the
        smallest one since any other set $Z \subsetneq \bigcup P$ is missing
        some element that is in one of the elements of $P$.
    \end{itemize}
  \end{proof}

  \begin{theorem}[Monotonicity]
    \label{thm:sem-mono}
    $\forall \; C \in \lang$ $\sem{C}$ is monotone.
  \end{theorem}
  \begin{proof}
    We want to prove that $\forall P, Q \in \pow{\states}$ and $C \in \lang$
    $$P \subseteq Q \implies \sem{C}(P) \subseteq \sem{C}(Q)$$
    By structural induction on $C$:
    \begin{itemize}
      \item $\sskip$:
        \begin{align*}
          \sem{\sskip}(P) 
            &= P & \text{By definition of $\sem{\sskip}$}\\
            &\subseteq Q \\
            &= \sem{\sskip}(Q) & \text{By definition of $\sem{\sskip}$}\\
        \end{align*}

      \item $b$:
        \begin{align*}
          \sem{b}(P) 
            &= \{ \bsem{b}(x)\downarrow \; \mid x \in P \} 
            & \text{By definition of $\sem{b}$}\\
            &\subseteq \{ \bsem{b}(x)\downarrow \; \mid x \in Q \} 
            & \text{Since $P \subseteq Q$}\\
            &= \sem{b}(Q) & \text{By definition of $\sem{b}$}\\
        \end{align*}

      \item $C_1 \fcmp C_2$:

        By inductive hypothesis $\sem{C_1}$ is monotone hence
        $\sem{C_1}(P) \subseteq \sem{C_2}(Q)$

        \begin{align*}
          \sem{C_1 \fcmp C_2}(P) 
            &= \sem{C_2}(\sem{C_1}(P))
            &\text{By definition of $\sem{C_1 \fcmp C_2}$}\\
            &\subseteq \sem{C_2}(\sem{C_1}(Q))
            &\text{By inductive hypothesis on $\sem{C_2}$} \\
        \end{align*}
    
      \item $C_1 + C_2$:
        \begin{align*}
          \sem{C_1 + C_2}(P) 
            &= \sem{C_1}(P) \cup \sem{C_2}(P)
            &\text{By definition of $\sem{C_1 + C_2}$}\\
            &\subseteq \sem{C_1}(Q) \cup \sem{C_2}(P)
            &\text{By inductive hypothesis on $\sem{C_1}$} \\
            &\subseteq \sem{C_1}(Q) \cup \sem{C_2}(Q)
            &\text{By inductive hypothesis on $\sem{C_2}$} \\
            &= \sem{C_1 + C_2}(Q) 
            &\text{By definition of $\sem{C_1 + C_2}$}\\
        \end{align*}
      
      \item $C^\fix$:

        \begin{align*}
          \sem{C^\fix}(P) 
            &= lfp(\lambda P'. P \cup \sem{C}(P'))
            &\text{By definition of $\sem{C^\fix}$}\\
            &\subseteq lfp(\lambda P'. Q \cup \sem{C}(P'))
            &\text{By theorem \ref{thm:lfp-mono}}\\
            &= \sem{C^\fix}(Q) 
            &\text{By definition of $\sem{C^\fix}$}\\
        \end{align*}

    \end{itemize}

  \end{proof}

  \begin{lemma}[$\sem{\cdot}$ well-defined]
    $\forall \; C \in \lang$ $\sem{C}$ is well-defined.
  \end{lemma}
  \begin{proof}
    From theorems \ref{thm:sem-lattice}, \ref{thm:sem-mono} and 
    \ref{thm:knaster} all the least fixpoints in the definition of 
    $\sem{C^\fix}$ exists; for all the other commands the semantics is 
    trivially well-defined.
  \end{proof}

  \begin{observation}
    As observed in \cite{Fischer79} when the set of base commands contains a 
    command to discard executions we can define the usual deterministic control 
    flow commands as syntactic sugar.

    $$if \; b \; then \; C_1 \; else \; C_2 \defeq (b ? \fcmp C_1) 
    + (\neg b ? \fcmp C_2)$$
    $$while \; b \; do \; C \defeq (b? \fcmp C)^\fix \fcmp \neg b ?$$
  \end{observation}

  \begin{observation}
    Some other languages usually provide an iteration command usually denoted
    $C^\star$ whose semantics is $\sem{C^\star}(P) \defeq \bigcup_{n \in \nat}
    \sem{C}^n(P)$, this is equivalent to $C^\fix$, the reasoning on why a
    fixpoint formulation was chosen will become clear in 
    \ref{obs:abstract-fix}.
  \end{observation}

\end{definition}
