\section{Abstract Hoare Logic}
\label{chp:intro-ahorare}

\subsection{Hoare logic}
Hoare logic was the first program logic ever designed by Hoare and Floyd 
\cite{Hoare69, Floyd93} and is based on the core concept of partial correctness 
assertions. A triple is a formula $\htriple{P}{C}{Q}$ where $P$ and $Q$ are 
assertions on the initial and final states of running program $C$, respectively. 
These assertions are partial in the sense that $Q$ is meaningful only when the 
execution of $C$ terminates.

Hoare logic is organized as a proof system, where the syntax 
$\vdash \htriple{P}{C}{Q}$ indicates that the triple 
$\htriple{P}{C}{Q}$ is proved by some combination of rules of the proof system.

The original formulation of Hoare logic was given for an imperative language 
with imperative constructs, but it can be easily translated for our language 
$\lang$ following the work in \cite{Moller21}.

\begin{definition}[Hoare triple]
  \label{def:hoare}
  Fixed the semantics of the base commands, an Hoare triple written 
  $\htriple{P}{C}{Q}$ is valid if and only if $\sem{C}(P) \subseteq Q$.

  $$\htriple{P}{C}{Q} \iff \sem{C}(P) \subseteq Q$$
\end{definition}

\begin{definition}[Hoare logic]$\;$ \\
  \label{def:hoaretules}
  \begin{prooftree}
    \AxiomC{$ $}
    \RightLabel{$(\sskip)$}
    \UnaryInfC{$\vdash \htriple{P}{\sskip}{P}$}
  \end{prooftree}

  % Rule for a basic command
  \begin{prooftree}
    \AxiomC{$ $}
    \RightLabel{$(base)$}
    \UnaryInfC{$\vdash \htriple{P}{b}{\bsem{b}(P)}$}
  \end{prooftree}

  % Rule for sequential composition
  \begin{prooftree}
    \AxiomC{$\vdash \htriple{P}{C_1}{Q}$}
    \AxiomC{$\vdash \htriple{Q}{C_2}{R}$}
    \RightLabel{$(seq)$}
    \BinaryInfC{$\vdash \htriple{P}{C_1 \fcmp C_2}{R}$}
  \end{prooftree}

  % Rule for nondeterministic choice
  \begin{prooftree}
    \AxiomC{$\vdash \htriple{P}{C_1}{Q}$}
    \AxiomC{$\vdash \htriple{P}{C_2}{Q}$}
    \RightLabel{$(disj)$}
    \BinaryInfC{$\vdash \htriple{P}{C_1 + C_2}{Q}$}
  \end{prooftree}

  \begin{prooftree}
    \AxiomC{$\vdash \htriple{P}{C}{P}$}
    \RightLabel{$(iterate)$}
    \UnaryInfC{$\vdash \htriple{P}{C^\fix}{P}$}
  \end{prooftree}

  % Rule for strengthening the precondition and weakening the postcondition
  \begin{prooftree}
    \AxiomC{$P \subseteq P'$}
    \AxiomC{$\vdash \htriple{P'}{C}{Q'}$}
    \AxiomC{$Q' \subseteq Q$}
    \RightLabel{$(consequence)$}
    \TrinaryInfC{$\vdash \htriple{P}{C}{Q}$}
  \end{prooftree}
\end{definition}

The proof system described in Definition \ref{def:hoaretules} is logically 
sound, meaning that all the triples provable by it are valid with respect to 
the definition in \ref{def:hoare}. This result was already present in the 
original work \cite{Hoare69}.

\begin{theorem}[Soundness]
  $$\vdash \htriple{P}{C}{Q} \implies \htriple{P}{C}{Q}$$
\end{theorem}

As observed by Cook in \cite{Cook78}, the reverse implication is not true in 
general, as a consequence of Gödel's incompleteness theorem. For this reason, 
Cook developed the concept of relative completeness, in which all instances of 
$\subseteq$ are provided by an oracle, proving that the incompleteness of the 
proof system is only caused by the incompleteness of the assertion language.

\begin{theorem}[Relative completeness]
  \label{thm:hlogic-complete}
  $$\htriple{P}{C}{Q} \implies \vdash \htriple{P}{C}{Q}$$
\end{theorem}

\subsection{Abstracting Hoare logic}
The idea of developing a Hoare-like logic to reason about properties of 
programs expressible within the theory of lattices using concepts from abstract 
interpretation is not new. In fact, \cite{Cousot12} already proposed a framework 
to perform this kind of reasoning. However, the validity of such triples is 
dependent on the standard definition of Hoare triples, and the proof system 
provided is incomplete if we ignore the rule to embed standard Hoare triples 
in the abstract ones.

Our approach will be different. In particular, the meaning of abstract Hoare 
triples will be dependent on the abstract inductive semantics, and we will 
provide a sound and (relatively) complete proof system that fully operates in 
the abstract.

\begin{definition}[Abstract Hoare triple]
  \label{def:aht}
  Given an abstract inductive semantics $\asem{\cdot}$ on the complete lattice
  $A$, the abstract Hoare triple written $\atriple{P}{C}{Q}$ is valid if
  and only if $\asem{C}(P) \leq_A Q$.

  $$\atriple{P}{C}{Q} \iff \asem{C}(P) \leq_A Q$$
\end{definition}

The definition is equivalent as the one provided in definition \ref{def:hoare} 
but here the abstract inductive semantics is used to procide the strongest 
postcondition of programs.

\subsubsection{Proof system}
As per Hoare logic we will peovide a sound an relatively complete (in the sense
of \cite{Cook78}) proof system to derive valid abstract Hoare triples in a 
compositional manner.

\begin{definition}[Abstract Hoare rules]$\;$\\
  \label{def:ahtrules}
  % Rule for the identity command
  \begin{prooftree}
    \AxiomC{$ $}
    \RightLabel{$(\sskip)$}
    \UnaryInfC{$\vdash \atriple{P}{\sskip}{P}$}
  \end{prooftree}
  The identity command does not change the state, so if $P$ holds before,
  it will hold after the execution.

  % Rule for a basic command
  \begin{prooftree}
    \AxiomC{$ $}
    \RightLabel{$(b)$}
    \UnaryInfC{$\vdash \atriple{P}{b}{\bsem{b}^A(P)}$}
  \end{prooftree}
  For a basic command $b$, if $P$ holds before the execution, then 
  $\bsem{b}^A(P)$ holds after the execution.

  % Rule for sequential composition
  \begin{prooftree}
    \AxiomC{$\vdash \atriple{P}{C_1}{Q}$}
    \AxiomC{$\vdash \atriple{Q}{C_2}{R}$}
    \RightLabel{$(\mathbb{\fcmp})$}
    \BinaryInfC{$\vdash \atriple{P}{C_1 \fcmp C_2}{R}$}
  \end{prooftree}
  If executing $C_1$ from state $P$ leads to state $Q$, and executing $C_2$
  from state $Q$ leads to state $R$, then executing $C_1$ followed by $C_2$
  from state $P$ leads to state $R$.

  % Rule for nondeterministic choice
  \begin{prooftree}
    \AxiomC{$\vdash \atriple{P}{C_1}{Q}$}
    \AxiomC{$\vdash \atriple{P}{C_2}{Q}$}
    \RightLabel{$(+)$}
    \BinaryInfC{$\vdash \atriple{P}{C_1 + C_2}{Q}$}
  \end{prooftree}
  If executing either $C_1$ or $C_2$ from state $P$ leads to state $Q$, 
  then executing the nondeterministic choice $C_1 + C_2$ from state $P$
  also leads to state $Q$.

  % Rule for iteration (Kleene star)
  \begin{prooftree}
    \AxiomC{$\vdash \atriple{P}{C}{P}$}
    \RightLabel{$(\fix)$}
    \UnaryInfC{$\vdash \atriple{P}{C^\fix}{P}$}
  \end{prooftree}
  If executing command $C$ from state $P$ leads back to state $P$, then 
  executing $C$ repeatedly (zero or more times) from state $P$ also leads
  back to state $P$.

  % Rule for strengthening the precondition and weakening the postcondition
  \begin{prooftree}
    \AxiomC{$P \leq P'$}
    \AxiomC{$\vdash \atriple{P'}{C}{Q'}$}
    \AxiomC{$Q' \leq Q$}
    \RightLabel{$(\leq)$}
    \TrinaryInfC{$\vdash \atriple{P}{C}{Q}$}
  \end{prooftree}
  If $P$ is stronger than $P'$ and $Q'$ is stronger than $Q$, then we can
  derive $\atriple{P}{C}{Q}$ from $\atriple{P'}{C}{Q'}$.
\end{definition}

The proofsystem in nonother than the proofsystem of definition \ref{def:hoaretules}
where the assertion are replaced by elements of the complete lattice $A$.

Note that we denonte abstract hoare triples as defined in defintion \ref{def:aht}
with the notation $\atriple{P}{C}{Q}$ and intread we denote the triples obtained
with the inference rules of definition \ref{def:ahtrules} with $\vdash 
\atriple{P}{C}{Q}$.

The proofsystem is sound:
\begin{theorem}[Soundness]
  \label{thm:atriple-sound}
  $$\vdash \atriple{P}{C}{Q} \implies \atriple{P}{C}{Q}$$
\end{theorem}
\begin{proof}
  By structural induction on the last rule applied in the derivation of
  $\vdash \atriple{P}{C}{Q}$:
  \begin{itemize}

    \item $(\sskip)$:
      Then the last step in the derivation was: 
      \begin{prooftree}
        \AxiomC{$ $}
        \RightLabel{$(\sskip)$}
        \UnaryInfC{$\vdash 
          \atriple{P}{\sskip}{P}$}
      \end{prooftree}

      The triple is valid since:
      \begin{align*}
        \asem{\sskip}(P)
          &= P &\text{By definition of $\asem{\cdot}$}
      \end{align*}

      \item $(b)$:
        Then the last step in the derivation was:
        \begin{prooftree}
          \AxiomC{$ $}
          \RightLabel{$(b)$}
          \UnaryInfC{$\vdash 
          \atriple{P}{b}{\bsem{b}^A(P)}$}
        \end{prooftree}

        The triple is valid since:
        \begin{align*}
          \asem{b}(P)
            &= \bsem{b}^A(P)
            & \text{By definition of $\asem{\cdot}$}
        \end{align*}

      \item $(\fcmp)$: Then the last step in the derivation was:
        \begin{prooftree}
          \AxiomC{$\vdash \atriple{P}{C_1}{Q}$}
          \AxiomC{$\vdash \atriple{Q}{C_2}{R}$}
          \RightLabel{$(\mathbb{\fcmp})$}
          \BinaryInfC{$\vdash \atriple{P}{C_1 \fcmp C_2}
            {R}$}
        \end{prooftree}
          
        By inductive hypothesis:
        $\asem{C_1}(P) \leq_A Q$ and
        $\asem{C_2}(Q) \leq_A R$.

        The triple is valid since:
        \begin{align*}
          \asem{C_1 \fcmp C_2}(P)
            &= \asem{C_2}(\asem{C_1}(P))
            &\text{By definition of $\asem{\cdot}$} \\
            &\leq_A \asem{C_2}(Q)
            &\text{By monotonicity of $\asem{\cdot}$} \\
            &\leq_A R
        \end{align*}

      \item $(+)$: Then the last step in the derivation was:
        \begin{prooftree}
          \AxiomC{$\vdash \atriple{P}{C_1}{Q}$}
          \AxiomC{$\vdash \atriple{P}{C_2}{Q}$}
          \RightLabel{$(+)$}
          \BinaryInfC{$\vdash \atriple{P}{C_1 + C_2}{Q}$}
        \end{prooftree}

        By inductive hypothesis: $\asem{C_1}(P) \leq Q$ and
        $\asem{C_2}(P) \leq Q$.

        The triple is valid since:
        \begin{align*}
          \asem{C_1 + C_2}(P)
            &= \asem{C_1}(P) \join_A \asem{C_2}(P)
            &\text{By definition of $\asem{\cdot}$} \\
            &\leq_A Q \join_A Q \\
            &= Q
        \end{align*}

      \item $(\fix)$:
        Then the last step in the derivation was:
        \begin{prooftree}
          \AxiomC{$\vdash \atriple{P}{C}{P}$}
          \RightLabel{$(\fix)$}
          \UnaryInfC{$\vdash \atriple{P}{C ^ \lfp}{P}$}
        \end{prooftree}

        By inductive hypothesis: $\asem{C}P \leq P$

        \begin{align*}
          \bsem{C^\fix}(P)
            &= \lfp(\lambda P' \to P \join_A \asem{C}(P')) \\
        \end{align*}

        We will show that $P$ is a fixpoint of 
        $\lambda P' \to P \join_A \asem{C}(P')$.

        \begin{align*}
          (\lambda P' \to P \join_A \asem{C}(P'))(P)
            &= P \join_A \asem{C}(P)
            & \text{since $\asem{C}(P) \leq P$} \\
            &= P
        \end{align*}

        Hence $P$ is a fixpoint of $\lambda P' \to P \join_A \asem{C}(P')$.

        And clearly is bigger than the least one 
        $\lfp(\lambda P' \to P \join_A \asem{C}(P')) \leq_A P$ thus making the
        triple valid.

      \item $(\leq)$: Then the last step in the derivation was:
        \begin{prooftree}
          \AxiomC{$P \leq P'$}
          \AxiomC{$\vdash \atriple{P'}{C}{Q'}$}
          \AxiomC{$Q' \leq Q$}
          \RightLabel{$(\leq)$}
          \TrinaryInfC{$\vdash \atriple{P}{C}{Q}$}
        \end{prooftree}

        By inductive hypothesis: $\asem{C}(P') \leq_A Q'$.
        
        The triple is valid since:
        \begin{align*}
          \asem{C}(P)
            & \asem{C}(P')
            & \text{By monotonicity of $\asem{\cdot}$}\\
            & \leq_A Q' & \text{By inductive hypothesis} \\
            & \leq_A Q
        \end{align*}
  \end{itemize}
\end{proof}

And is also relatively complete, in the sense that the axioms are complete 
relative to what we can prove in the underlying assertion language, that in
our case is described by the complete lattice.

We will start by proving a slightly weaker result:

\begin{theorem}[Relative $\asem{\cdot}$-completeness]
  \label{thm:post-completeness}
  $$\vdash \atriple{P}{C}{\asem{C}(P)}$$
\end{theorem}
\begin{proof}
  By structural induction on $C$:
  \begin{itemize}

    \item $\sskip$:
      By definition $\asem{\sskip}(P) = P$
      \begin{prooftree}
        \AxiomC{$ $}
        \RightLabel{$(\sskip)$}
        \UnaryInfC{$\vdash \atriple{P}{\sskip}{P}$}
      \end{prooftree}

      \item $b$:
        By definition $\asem{b}(P) = \bsem{b}^A(P)$
        \begin{prooftree}
          \AxiomC{$ $}
          \RightLabel{$(b)$}
          \UnaryInfC{$\vdash \atriple{P}{b}{\bsem{b}^A(P)}$}
        \end{prooftree}

      \item $C_1 \fcmp C_2$:
        By definition $\asem{C_1 \fcmp C_2}(P) = 
        \asem{C_2}(\asem{C_1}(P))$

        \begin{prooftree}
          \AxiomC{(Inductive hypothesis)}
          \noLine
          \UnaryInfC{$\vdash \atriple{P}{C_1}{\asem{C_1}(P)}$}
          \AxiomC{(Inductive hypothesis)}
          \noLine
          \UnaryInfC{$\vdash \atriple{\asem{C_1}(P)}{C_2}
            {\asem{C_2}(\asem{C_1}(P))}$}
          \RightLabel{$(\fcmp)$}
          \BinaryInfC{$\vdash \atriple{P}{C_1 \fcmp C_2}
            {\asem{C_2}(\asem{C_1}(P))}$}
        \end{prooftree}


      \item $C_1 + C_2$:
        By definition $\bsem{C_1 + C_2}(P) = 
        \bsem{C_1}(P) \join_A \bsem{C_2}(P)$

        \begin{prooftree}
          \AxiomC{$\pi_1$}
          \AxiomC{$\pi_2$}
          \RightLabel{$(+)$}
          \BinaryInfC{$\vdash \atriple{P}{C_1 + C_2}
            {\asem{C_1}(P) \join_A \asem{C_2}(P)}$}
        \end{prooftree}

        Where $\pi_1$:
        \begin{prooftree}
          \AxiomC{$P \leq_A P$}
          \AxiomC{(Inductive hypothesis)}
          \noLine
          \UnaryInfC{$\vdash \atriple{P}{C_1}{\asem{C_1}(P)}$}
          \AxiomC{$\asem{C_1}(P) \leq_A \asem{C_1}(P) \join_A \asem{C_2}(P)$}
          \RightLabel{$(\leq)$}
          \TrinaryInfC{$\vdash \atriple{P}{C_1}
            {\asem{C_1}(P) \join_A \asem{C_2}(P)}$}
        \end{prooftree}

        and $\pi_2$:
        \begin{prooftree}
          \AxiomC{$P \leq_A P$}
          \AxiomC{(Inductive hypothesis)}
          \noLine
          \UnaryInfC{$\vdash \atriple{P}{C_2}{\asem{C_2}(P)}$}
          \AxiomC{$\asem{C_2}(P) \leq_A \asem{C_1}(P) \join_A \asem{C_2}(P)$}
          \RightLabel{$(\leq)$}
          \TrinaryInfC{$\vdash \atriple{P}{C_2}
            {\asem{C_1}(P) \join_A \asem{C_2}(P)}$}
        \end{prooftree}

      \item $C^\fix$:
        By definition $\bsem{C^\fix}(P) = lfp(\lambda P' \to P \join_A
        \asem{C}(S')$.

        Let $K \defeq lfp(\lambda P' \to P \join_A \asem{C}(S')$
        hence $K = P \join_A \asem{C}(K)$ since it is a fixpoint, thus
        \begin{itemize}
          \item $\alpha_1$: $K \geq_A P$
          \item $\alpha_2$: $K \geq_A \asem{C}(K)$
        \end{itemize}

          \begin{prooftree}
            \AxiomC{$\alpha_1$}
            \AxiomC{$K \leq_A K$}
            \AxiomC{(Inductive hypothesis)}
            \noLine
            \UnaryInfC{$\vdash \atriple{K}{C}{\asem{C}(K)}$}
            \AxiomC{$\alpha_2$}
            \TrinaryInfC{$\vdash \atriple{K}{C}{K}$}
            \RightLabel{$(\fix)$}
            \UnaryInfC{$\vdash \atriple{K}{C^\fix}{K}$}
            \AxiomC{$K \leq_A K$}
            \RightLabel{$(\leq)$}
            \TrinaryInfC{$\vdash \atriple{P}{C^\fix}{K}$}
          \end{prooftree}
  \end{itemize}
\end{proof}

Now we can finally show the relative completeness:
\begin{theorem}[Relative completeness]
  \label{thm:completeness}
  $$\atriple{P}{C}{Q} \implies \vdash \atriple{P}{C}{Q}$$
\end{theorem}
\begin{proof}
  By definition of $\atriple{P}{C}{Q} \iff Q \geq_A \asem{C}(P)$

  \begin{prooftree}
    \AxiomC{$P \leq_A P$}
    \AxiomC{(By Theorem \ref{thm:post-completeness})}
    \noLine
    \UnaryInfC{$\vdash \atriple{P}{C}{\asem{C}(P)}$}
    \AxiomC{$Q \geq_A \asem{C}(P)$}
    \RightLabel{$(\leq)$}
    \TrinaryInfC{$\vdash \atriple{P}{C}{Q}$}
  \end{prooftree}
\end{proof}
