\chapter{Instantiating Abstract Hoare Logic}

In this chapter, we will show how to instantiate abstract Hoare logic to create 
new proof systems. We will also demonstrate that the framework of abstract Hoare 
logic is so general that, in some instantiations, it is able to reason about 
properties that are not expressible in standard Hoare logic.

\subimport{./}{hoare}
\subimport{./}{interval}
\subimport{./}{hyper_hoare}

\section{Partial Incorrectness}

Any instantiation of the abstract inductive semantics gives us another
instantiation for free, since the semantics is parametric on a complete lattice
$A$, and the dual of a complete lattice is also a complete lattice. We can
obtain the dual abstract inductive semantics for free on the complete lattice
$A^{op}$.

\begin{definition}[Dual Abstract Inductive Semantics]
  Given an abstract inductive semantics defined on some complete lattice $A$
  and base commands $\bsem{\cdot}^A$, we can define the dual abstract inductive
  semantics as the abstract inductive semantics instantiated on the 
  complete lattice $A^{op}$ with base command semantics $\bsem{\cdot}^{A^{op}}
  = \bsem{\cdot}^A$.
\end{definition}

Since the dual abstract inductive semantics is an abstract inductive semantics,
it also induces an Abstract Hoare Logic. In the dual lattice, since the partial
order is inverted, the roles of joins and meets are inverted, and thus also
$\lfp$ and $\gfp$ are inverted. Hence, the dual abstract inductive semantics,
when seen from the dual lattice, can be expressed as: 
\begin{align*}
  \asem[A^{op}]{\sskip}        &= id &&= id \\
  \asem[A^{op}]{b}             &= \bsem{b}^{A^{op}} &&= \bsem{b}^A \\
  \asem[A^{op}]{C_1 \fcmp C_2} &= \asem[A^{op}]{C_2} \circ \asem[A^{op}]{C_1} &&= \asem[A^{op}]{C_2} \circ \asem[A^{op}]{C_1} \\
  \asem[A^{op}]{C_1 + C_2}     &= \lambda P . \asem[A^{op}]{C_1} P \join_{A^{op}} \asem[A^{op}]{C_2} P &&= \lambda P . \asem[A^{op}]{C_1} P \meet_A \asem[A^{op}]{C_2} P \\
  \asem[A^{op}]{C^\fix}        &= \lambda P . \lfp_{A^{op}}(\lambda P'. P \join_{A^{op}} \asem[A^{op}]{C} P') &&= \lambda P . \gfp_{A}(\lambda P'. P \meet_A \asem[A^{op}]{C} P')
\end{align*}

Following the intuition that the abstract inductive semantics is some abstract
version of the strongest postcondition, what interpretation can we give for the
dual abstract inductive semantics? While on the lattice $A$, for the
non-deterministic choice, we take the join of the two branches. In the opposite
lattice, we take their meet. Following this intuition, instead of taking all
the reachable states (the union of the states reached by the two branches), we
take the states that we are sure we are reaching, i.e., the intersection of the
states reached by the two branches. The same reasoning also holds for the
$\fix$ command.

Since the order on the dual lattice is inverted, inversion holds also for the
validity of the Abstract Hoare triples: $$\models \atriple[A^{op}]{P}{C}{Q}
\iff \asem{C}(P) \leq_{A^{op}} Q \iff \asem[A^{op}]{C}(P) \geq_A Q$$

When the dual abstract inductive semantics is obtained with the abstract
inductive semantics on $\pow{\states}$ (the strongest postcondition), the dual
semantics becomes the strongest liberal postcondition introduced in
\cite{Zhang22} (in the boolean case). The triples are given the name "partial
incorrectness," as if $\models \atriple[A^{op}]{Q}{C}{P}$, $P$ is an
over-approximation of the states that end up in $Q$ modulo termination. The
same concept is studied under the name of "Necessary Preconditions" in
\cite{Cousot13}, and thanks to Abstract Hoare Logic, we obtained a sound and
complete proof system for the logic.

