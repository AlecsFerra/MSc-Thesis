\section{Interval Logic and Algebraic Hoare Logic}

\subsection{Algebraic Hoare Logic}

As explained in section \ref{chp:intro-ahorare}, Abstract Hoare Logic was
inspired by Algebraic Hoare Logic \cite{Cousot12}. Both logics can be used to
prove properties chosen in computer-representable abstract domains.

\begin{definition}[Algebraic Hoare triple]
  Given two Galois insertions $\langle \pow{\states}, \subseteq \rangle
  \galoiS{\alpha_1}{\gamma_1} \langle A, \leq \rangle$ and $\langle
  \pow{\states}, \subseteq \rangle \galoiS{\alpha_2}{\gamma_2} \langle B,
  \sqsubseteq \rangle$, an Algebraic Hoare triple written $\ctriple{P}{C}{Q}$ is
  valid if and only if $\htriple{\gamma_1(P)}{C}{\gamma_2(Q)}$ is valid.
  $$\ctriple{P}{C}{Q} \iff \htriple{\gamma_1(P)}{C}{\gamma_2(Q)}$$
\end{definition}

From this definition, we see that the definition of Algebraic Hoare Logic is
deeply connected to standard Hoare Logic and thus to the strongest postcondition
of the program in the concrete domain.

\begin{definition}[Algebraic Hoare logic proof system\footnote{Rules
  $(\overline{\join})$ and $(\overline{\meet})$ are missing but will be
  discussed in section \ref{chp:join-meet-rules}}]$\;$\\
  \begin{prooftree}
    \AxiomC{$ $}
    \RightLabel{$(\overline{\bot})$}
    \UnaryInfC{$\ctriple{\bot_1}{C}{Q}$}
  \end{prooftree}

  \begin{prooftree}
    \AxiomC{$ $}
    \RightLabel{$(\overline{\top})$}
    \UnaryInfC{$\ctriple{P}{C}{\top_2}$}
  \end{prooftree}
  
  \begin{prooftree}
    \AxiomC{$\htriple{\gamma_1(P)}{C}{\gamma_2(Q)}$}
    \RightLabel{$(\overline{S})$}
    \UnaryInfC{$\ctriple{P}{C}{Q}$}
  \end{prooftree}
  
  \begin{prooftree}
    \AxiomC{$P \leq P'$}
    \AxiomC{$\ctriple{P'}{C}{Q'}$}
    \AxiomC{$Q' \sqsubseteq Q$}
    \RightLabel{$(\overline{\implies})$}
    \TrinaryInfC{$\ctriple{P}{C}{Q}$}
  \end{prooftree}
\end{definition}

This proof system highlights that most of the work is done by rule
$(\overline{S})$, which embeds Hoare triples in Algebraic Hoare triples. One can
easily prove that the proof system is relatively complete from the relative
completeness of Hoare logic. In particular, only the $(\overline{S})$ rule is
actually needed, since all the implications in the abstract must also hold in
the concrete.

\subsection{Interval Logic} As shown in Definition \ref{def:aisgi}, via a Galois
insertion we can obtain a similar family of triples as those in Algebraic Hoare
Logic when the abstract domains used in the pre- and post-conditions are the
same.

\begin{example}[Interval logic]
  \label{exmp:int-logic}
  Applying Definition \ref{def:aisgi} to the Galois insertion on the interval
  domain defined in Example \ref{exmp:interval}, we obtain a sound and
  relatively complete logic to reason about properties of programs that are
  expressible as intervals.
\end{example}

\begin{example}[Derivation in interval logic]
  \label{exmp:int-deriv}
  Let $C \defeq ((x := 1) + (x := 3)) \fcmp (x = 2? \fcmp x := 5) + (x \neq 2?
  \fcmp x := x - 1)$

  Then the following derivation is valid:

  \begin{prooftree}
    \AxiomC{$\pi_1$}
    \AxiomC{$\pi_3$}
    \RightLabel{$(\fcmp)$}
    \BinaryInfC{$\vdash \atriple[Int]{\top}{C}{[0, 5]}$}
  \end{prooftree}

  $\pi_1$:
  \begin{prooftree}
    \AxiomC{$\top \leq \top$}
    \AxiomC{$$}
    \RightLabel{$(b)$}
    \UnaryInfC{$\vdash \atriple[Int]{\top}{x := 1}{[1, 1]}$}
    \AxiomC{$[1, 1] \leq [1, 3]$}
    \TrinaryInfC{$\vdash \atriple[Int]{\top}{x := 1}{[1, 3]}$}
    \AxiomC{$\pi_2$}
    \RightLabel{$(+)$}
    \BinaryInfC{$\vdash \atriple[Int]{\top}{(x := 1) + (x := 3)}{[1, 3]}$}
  \end{prooftree}

  $\pi_2$:
  \begin{prooftree}
    \AxiomC{$\top \leq \top$}
    \AxiomC{$$}
    \RightLabel{$(b)$}
    \UnaryInfC{$\vdash \atriple[Int]{\top}{x := 3}{[3, 3]}$}
    \AxiomC{$[3, 3] \leq [1, 3]$}
    \RightLabel{$(\leq)$}
    \TrinaryInfC{$\vdash \atriple[Int]{\top}{x := 3}{[1, 3]}$}
  \end{prooftree}

  $\pi_3$:
  \begin{prooftree}
    \AxiomC{$[1, 3] \leq [1, 3]$}
    \AxiomC{$$}
    \RightLabel{$(b)$}
    \UnaryInfC{$\vdash \atriple[Int]{[1, 3]}{x = 2?}{[2]}$}
    \AxiomC{$$}
    \RightLabel{$(b)$}
    \UnaryInfC{$\vdash \atriple[Int]{[2]}{x := 5}{[5]}$}
    \RightLabel{$(\fcmp)$}
    \BinaryInfC{$\vdash \atriple[Int]{[1, 3]}{x = 2? \fcmp x := 5}{[5]}$}
    \AxiomC{$[5, 5] \leq [0, 5]$}
    \TrinaryInfC{$\vdash \atriple[Int]{[1, 3]}{x = 2? \fcmp x := 5}{[0, 5]}$}
    \AxiomC{$\pi_4$}
    \RightLabel{$(+)$}
    \BinaryInfC{$\vdash \atriple[Int]{[1, 3]}{(x = 2? \fcmp x := 5) + (x \neq 2?
      \fcmp x := x - 1)}{[0, 5]}$}
  \end{prooftree}
  
  $\pi_4$:
  \begin{prooftree}
    \AxiomC{$[1, 3] \leq [1, 3]$}
    \AxiomC{$$}
    \RightLabel{$(b)$}
    \UnaryInfC{$\vdash \atriple[Int]{[1, 3]}{x \neq 2?}{[1, 3]}$}
    \AxiomC{$$}
    \RightLabel{$(b)$}
    \UnaryInfC{$\vdash \atriple[Int]{[1, 3]}{x := x - 1}{[0, 2]}$}
    \RightLabel{$(\fcmp)$}
    \BinaryInfC{$\vdash \atriple[Int]{[1, 3]}{x \neq 2? \fcmp x := x - 1}{[0,
      2]}$}
    \AxiomC{$[0, 2] \leq [0, 5]$}
    \TrinaryInfC{$\vdash \atriple[Int]{[1, 3]}{x \neq 2? \fcmp x := x - 1}{[0,
      5]}$}
  \end{prooftree}

  This is also the best we can derive since $\asem[Int]{C}(\top) = [0, 5]$.
\end{example}

\subsubsection{Applications}
This framework, like Algebraic Hoare Logic, can be used to specify how a static
analyzer for a given abstract domain should work. Since $\asem{\cdot}$ is the
best abstract analyzer on abstract domain $A$ when it is defined inductively,
and since the whole proof system is in the abstract, we can check that a
derivation is indeed correct algorithmically (as long as we can check
implications and base commands). These are usually the standard requirements for
an abstract domain to be useful. The same does not hold for Algebraic Hoare
Logic since deciding the validity of arbitrary triples would require deciding
the validity of standard Hoare logic triples, and in general, we cannot decide
implications between arbitrary properties.

\subsection{Relationship}
Clearly, Algebraic Hoare Logic can derive the same triples that are derivable by
Abstract Hoare Logic when instantiated through a Galois insertion from
$\pow{\states}$ as in Example \ref{exmp:int-logic}. From Theorem
\ref{thm:sound-ai}, $\asem{\cdot}$ is a sound overapproximation of
$\sem{\cdot}$.

\begin{theorem}
  $\atriple{P}{C}{Q} \implies \ctriple{P}{C}{Q}$
\end{theorem}
\begin{proof}
  \begin{align*}
    \atriple{P}{C}{Q}
      &\implies \asem{C}(P) \leq Q
      &\text{From Theorem \ref{thm:atriple-sound}} \\
      &\implies \sem{C}(\gamma(P)) \subseteq \gamma(Q)
      &\text{From Theorem \ref{thm:sound-ai}} \\
      &\implies \htriple{\gamma(P)}{C}{\gamma(Q)}
      &\text{From Theorem \ref{thm:hlogic-complete}} \\
      &\implies \ctriple{P}{C}{Q}
      &\text{From rule $(\overline{S})$} \\
  \end{align*}
\end{proof}

However, the converse is not true. The relative completeness of Algebraic Hoare
Logic is with respect to the best correct approximation of $\sem{\cdot}$ and not
with respect to $\asem{\cdot}$ as in Abstract Hoare Logic.

\begin{example}[Counter example]
  From Example \ref{exmp:int-deriv}, we know that $\atriple{\top}{C}{[0, 5]}$ is
  the best Abstract Hoare triple that we can obtain, but $\sem{C}{\top} = \{0,
  2\}$. Via Theorem \ref{thm:hlogic-complete}, we can obtain
  $\htriple{\top}{C}{\{0, 2\}}$. Hence, via the $(\overline{S})$ rule, we can
  obtain $\ctriple{\top}{C}{[0, 2]}$, which is unobtainable in Abstract Hoare
  Logic.
\end{example}
