\chapter{Conclusions}

\section{Extending the framework}
The following are only preliminary results that we weren't able to explore
because of time constraints.

\subsection{Under-approximating logics}
Hoare-style logics have not only been used to perform over-approximate
reasoning. Incorrectness logic \cite{OHearn19} uses the same framework as Hoare
logic but focuses on under-approximation. An incorrectness triple is defined as
$\htriple{P}{C}{Q} \iff \sem{C}(P) \supseteq Q$. The definition of easy
represents the same in Abstract Hoare logic. Validity differs from Hoare logic
in the direction of set inclusion, so we could easily provide the definition
for Abstract Incorrectness Logic triples $\atriple{P}{C}{Q} \iff \asem{C}(P)
\geq_A Q$ and perform the same translation that we did for the rules of Hoare
Logic for the rules of Incorrectness logic.

\subsection{Command composition and Weakest liberal precondition}
In the definition of the abstract inductive semantics, the meaning of command
composition is fixed $\asem{C_1 \fcmp C_2} \defeq \asem{C_2} \circ \asem{C_1}$.
This is generally an arbitrary assumption. In particular, if we interpret the
semantics of $\asem{\cdot}$ as the strongest postcondition on the lattice $A$,
the abstract inductive semantics on the opposite lattice $A^{op}$ with base
commands $\bsem{C}^{A^{op}} = (\bsem{C}^A)^{-1}$, then when the composition is
inverted, $\asem[A^{op}]{\cdot}$ is the weakest liberal precondition as $\join$
in the opposite lattice becomes $\meet$ and $\lfp$ becomes $\gfp$. Clearly, the
proof system needs to be modified in the composition rule that needs to be
inverted:

\begin{prooftree}
  \AxiomC{$\vdash \atriple{P}{C_2}{Q}$}
  \AxiomC{$\vdash \atriple{Q}{C_1}{R}$}
  \BinaryInfC{$\vdash \atriple{P}{C_1 \fcmp C_2}{R}$}
\end{prooftree}

All the other rules are unchanged. The system is still sound and complete. In
particular, we can note that from the definition $\atriple{Q}{C}{P} \iff
\asem[A^{op}]{Q} \leq P$, our triples now encode an over-approximation of the
weakest liberal precondition of program $C$ ending in the states $Q$.

These triples were already described in \cite{Cousot13} under the name of
necessary precondition, but no proof system was provided. By using this slight
modification of Abstract Hoare logic, we obtain a sound and complete proof
system for that.

\section{Related work}
The idea of systematically constructing program logics is not new. In chapter
\ref{chp:join-meet-rules}, we already discussed Kleene Algebras with Tests
\cite{Kozen97} and how the framework requires the non-deterministic choice to
distribute over command composition. This restricts the program semantics that
we can describe. Algebraic Hoare Logic proves that this requirement is
unnecessary to construct a sound and complete Hoare-style logic. Another
alternative proposed in \cite{Martin06} is built on the theory of traced
monoidal categories, where the monoidal structure of the category is used to
encode the semantics of the non-deterministic choice. This implicitly imposes
the same distributivity requirements of the non-deterministic choice as Kleene
Algebra with Tests. 
It is also important to note that the scope of all the
previous work is focused on relating the concrete semantics of programs to
concrete Hoare rules. Instead, this work aims to be more general, particularly
in how we are able to use the abstract inductive semantics to encode
under-approximations as abstract interpreters from abstract interpretation and
more complex properties as Hyper properties.

