\chapter{Conclusions}

We extendedn traditional Hoare logic into a more abstract and versatile 
framework. By incorporating principles from order theory and abstract 
interpretation, we developed a robust method for reasoning about a broader range 
of program properties. This abstract Hoare logic framework maintains the 
foundational principles of soundness and completeness while expanding its 
applicability.

We have shown that multiple program logics known in literature are actually 
special cases of Abstract Hoare logic, while constructing a program logic for 
hyperproperties in the framework of Abstract Hoare logic we also provided a 
compositional definition of strongest hyper postcondition.


Additionally, we have introduced the concept of Reverse Abstract Hoare logic, 
which enhances the framework by enabling backward reasoning by reusing the ideas
from Abstract Hoare logic. 

Throughout this work, we have demonstrated the practical applications of our 
framework through various examples and instantiations. These examples illustrate 
the versatility and effectiveness of the Abstract Hoare logic framework.

\section{Future work}
The following are only preliminary results that we weren't able to explore
because of time constraints.

\subsection{Under-approximating logics}
Hoare-style logics have not only been used to perform over-approximate
reasoning. Incorrectness logic \cite{OHearn19} uses the same framework as Hoare
logic but focuses on under-approximation. An incorrectness triple is defined as
$\htriple{P}{C}{Q} \iff \sem{C}(P) \supseteq Q$. The definition of easy
represents the same in Abstract Hoare logic. Validity differs from Hoare logic
in the direction of set inclusion, so we could easily provide the definition
for Abstract Incorrectness Logic triples $\atriple{P}{C}{Q} \iff \asem{C}(P)
\geq_A Q$ and perform the same translation that we did for the rules of Hoare
Logic for the rules of Incorrectness logic.

\subsection{Hyper domains}
We used hyper domains to construct program logics that that are able to reason
about hyperproperties, but we didn't explore what kind of calculi proofsystem 
they create when they are used to instantiate Reverse Hoare logic and on the 
opposite lattice.

\section{Related work}
The idea of systematically constructing program logics is not new. In chapter
\ref{chp:join-meet-rules}, we already discussed Kleene Algebras with Tests
\cite{Kozen97} and how the framework requires the non-deterministic choice to
distribute over command composition. This restricts the program semantics that
we can describe. Algebraic Hoare Logic proves that this requirement is
unnecessary to construct a sound and complete Hoare-style logic. Another
alternative proposed in \cite{Martin06} is built on the theory of traced
monoidal categories, where the monoidal structure of the category is used to
encode the semantics of the non-deterministic choice. This implicitly imposes
the same distributivity requirements of the non-deterministic choice as Kleene
Algebra with Tests. 
It is also important to note that the scope of all the
previous work is focused on relating the concrete semantics of programs to
concrete Hoare rules. Instead, this work aims to be more general, particularly
in how we are able to use the abstract inductive semantics to encode
under-approximations as abstract interpreters from abstract interpretation and
more complex properties as Hyper properties.

