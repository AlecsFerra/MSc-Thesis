\chapter{Backward Abstract Hoare Logic}

When defining the semantics for $\lang$, we implicitly assumed that the
abstract inductive semantics is forward by giving the definition $\asem{C_1
\fcmp C_2} \defeq \asem{C_2} \circ \asem{C_1}$. However, except for the rule
($\fcmp$), we never explicitly used this fact. We can reuse the theory of
Abstract Hoare logic to construct a slight variation called Backward Abstract
Hoare logic, which describes Hoare logics where the semantics is defined
backward.

\section{Framework}

\subsection{Backward abstract inductive semantics}

To define the backward version of Abstract Hoare logic, we first need a
backward version of the abstract inductive semantics:

\begin{definition}[Backward abstract inductive semantics]
  Given a complete lattice $A$ and a family of monotone functions $\bsem{b}^A :
  A \to A$ for all $b \in Base$, the abstract inductive semantics is defined as
  follows:

  \begin{align*}
      \asemback{\cdot}         & \;\;:\; \lang \to A \to A \\
      \asemback{\sskip}        &\defeq id \\
      \asemback{b}             &\defeq \bsem{b}^A \\
      \asemback{C_1 \fcmp C_2} &\defeq \asemback{C_1} \circ \asemback{C_2} \\
      \asemback{C_1 + C_2}     &\defeq \lambda P . \asemback{C_1} P \join_A \asemback{C_2} P \\
      \asemback{C^\fix}        &\defeq \lambda P . \lfp(\lambda P'. P \join_A \asemback{C} P')
  \end{align*}
\end{definition}

The only difference from the abstract inductive semantics provided in
definition \ref{def:abstract-inductive-semantics} is the case for $C_1 \fcmp
C_2$.

We can prove that the backward abstract inductive semantics is still monotone.

\begin{theorem}[Monotonicity]
  \label{thm:asem-mono-back} 
  For all $C \in \lang$, $\asemback{C}$ is monotone.
\end{theorem}

\begin{proof}
  We modify the inductive case of the proof of theorem \ref{thm:asem-mono} by
  providing only the case for $\asemback{C_1 \fcmp C_2}$ as all the other cases
  are identical.

  \begin{itemize}
    \item $C_1 \fcmp C_2$:

      By inductive hypothesis, $\asemback{C_2}$ is monotone, hence
      $\asemback{C_2}(P) \leq_A \asemback{C_2}(Q)$.

      \begin{align*}
        \asemback{C_1 \fcmp C_2}(P) 
          &= \asemback{C_1}(\asemback{C_2}(P))
          &\text{By definition of $\asemback{C_1 \fcmp C_2}$}\\
          &\leq_A \asemback{C_1}(\asemback{C_2}(Q))
          &\text{By inductive hypothesis on $\asemback{C_1}$} \\
      \end{align*}
  \end{itemize}
\end{proof}

\begin{lemma}[$\asemback{\cdot}$ well-defined]
  For all $C \in \lang$, $\asemback{C}$ is well-defined.
\end{lemma}
\begin{proof}
  From theorems \ref{thm:asem-mono-back} and \ref{thm:knaster}, all the least
  fixpoints in the definition of $\asemback{C^\fix}$ exist. For all the other
  commands, the semantics is trivially well-defined.
\end{proof}

\subsection{Backward Abstract Hoare Logic}

Now we can provide the definition for the backward abstract Hoare triples,
which is the same as for abstract Hoare triples, only with the backward
abstract inductive semantics instead of the usual abstract inductive semantics.

\begin{definition}[Backward Abstract Hoare triple]
  \label{def:baht}
  Given an abstract inductive semantics $\asemback{\cdot}$ on the complete
  lattice $A$, the abstract Hoare triple written $\atripleback{P}{C}{Q}$ is
  valid if and only if $\asemback{C}(P) \leq_A Q$.

  $$\atripleback{P}{C}{Q} \iff \asemback{C}(P) \leq_A Q$$
\end{definition}

Clearly, the proof system only needs to be modified to accommodate the new
semantics for program composition, and all the other rules remain the same.

\begin{definition}[Backward Abstract Hoare rules]$\;$\\
  We only provide the rule for program composition; all the other rules are
  identical to those provided in definition \ref{def:ahtrules}.

  % Rule for sequential composition
  \begin{prooftree}
    \AxiomC{$\vdash \atripleback{P}{C_2}{Q}$}
    \AxiomC{$\vdash \atripleback{Q}{C_1}{R}$}
    \RightLabel{$(\mathbb{\fcmp})$}
    \BinaryInfC{$\vdash \atripleback{P}{C_1 \fcmp C_2}{R}$}
  \end{prooftree}
  If executing $C_2$ from state $P$ leads to state $Q$, and executing $C_1$
  from state $Q$ leads to state $R$, then executing $C_2$ followed by $C_1$ from
  state $P$ leads to state $R$. 
\end{definition}

We can prove again that the proof system is still sound and complete.

\begin{theorem}[Soundness]
  \label{thm:atriple-sound-back}
  $$\vdash \atripleback{P}{C}{Q} \implies \atripleback{P}{C}{Q}$$
\end{theorem}

\begin{proof}
  We modify the inductive case of the proof of theorem \ref{thm:atriple-sound}
  by providing only the case for rule $(\fcmp)$ as all the other cases are
  identical.

  \begin{itemize}
      \item $(\fcmp)$: Then the last step in the derivation was:
        \begin{prooftree}
          \AxiomC{$\vdash \atripleback{P}{C_2}{Q}$}
          \AxiomC{$\vdash \atripleback{Q}{C_1}{R}$}
          \RightLabel{$(\mathbb{\fcmp})$}
          \BinaryInfC{$\vdash \atripleback{P}{C_1 \fcmp C_2}{R}$}
        \end{prooftree}
          
        By inductive hypothesis:
        $\asemback{C_2}(P) \leq_A Q$ and
        $\asemback{C_1}(Q) \leq_A R$.

        The triple is valid since:
        \begin{align*}
          \asemback{C_1 \fcmp C_2}(P)
            &= \asemback{C_1}(\asemback{C_2}(P))
            &\text{By definition of $\asemback{\cdot}$} \\
            &\leq_A \asemback{C_1}(Q)
            &\text{By monotonicity of $\asemback{\cdot}$} \\
            &\leq_A R
        \end{align*}
  \end{itemize}
\end{proof}

\begin{theorem}[Relative $\asemback{\cdot}$-completeness]
  \label{thm:post-completeness-back}
  $$\vdash \atripleback{P}{C}{\asemback{C}(P)}$$
\end{theorem}

\begin{proof}
  We modify the inductive case of the proof of theorem
  \ref{thm:post-completeness} by providing only the case for $C_1 \fcmp C_2$ as
  all the other cases are identical.

  \begin{itemize}
      \item $C_1 \fcmp C_2$:
        By definition $\asemback{C_1 \fcmp C_2}(P) = \asemback{C_1}(\asemback{C_2}(P))$

        \begin{prooftree}
          \AxiomC{(Inductive hypothesis)}
          \noLine
          \UnaryInfC{$\vdash \atripleback{P}{C_2}{\asemback{C_2}(P)}$}
          \AxiomC{(Inductive hypothesis)}
          \noLine
          \UnaryInfC{$\vdash \atripleback{\asemback{C_2}(P)}{C_1}{\asemback{C_1}(\asemback{C_2}(P))}$}
          \RightLabel{($\fcmp$)}
          \BinaryInfC{$\vdash \atripleback{P}{C_1 \fcmp C_2}{\asemback{C_1 \fcmp C_2}(P)}$}
        \end{prooftree}
  \end{itemize}
\end{proof}


\subsection{Backward-Forward Abstract Inductive Semantics Duality}

\begin{definition}[Dual semantics]
  Given an abstract inductive semantics defined on some lattice $A$ with
  base commands semantics $\bsem{\cdot}^A$, we can define the dual backward 
  abstract inductive semantics as the backward abstract inductive semantics
  instantiated on the lattice $A^{op}$ where the semantics of the
  base commands is defined as $\bsem{\cdot}^{A^{op}} = (\bsem{\cdot}^A)^{-1}$.
\end{definition}

If we interpret the abstract inductive semantics as the strongest postcondition
on the domain $A$, then the backward abstract inductive semantics on the domain 
$A$ is the weakest precondition on the domain $A$, this is true since on the 
opposite lattice the role of meets and joins i inverted ($\join_A^{op} = 
\meet_A$ and $\lfp_{A^{op}} = \gfp_A$).

\begin{observation}
  \label{obs:weakest-precondition}
  Following observation \ref{obs:weakest-precondition}, the fual backward 
  abstract inductive semantics on the domain $\pow{\states}^{op}$ is the weakest 
  liberal precondition.
\end{observation}

\section{Instantiations}

\subsection{Hoare Logic, Again}

From the instantiations of the backward abstract inductive semantics of 
observation \ref{obs:weakest-precondition}, we know that 
$\asemback[\pow{\states}^{op}]{\cdot}$ is the weakest liberal precondition. By 
looking at the meaning of the backward abstract Hoare triples:

\begin{theorem}[Dual-logic equivalence]
  $$\atripleback[\pow{\states}^{op}]{Q}{C}{P} \iff \atriple{P}{C}{Q}$$
\end{theorem}
\begin{proof}
\begin{align*}
  \atripleback[\pow{\states}^{op}]{Q}{C}{P}
    &\iff \asemback[\pow{\states}^{op}]{C}(Q) \leq_{\pow{\states}^{op}} P \\
    &\iff wlp(C, Q) \leq_{\pow{\states}^{op}} P \\
    &\iff wlp(C, Q) \geq_{\pow{\states}} P \\
    &\iff stp(C, P) \leq_{\pow{\states}} Q \\
    &\iff \asem{C}(P) \leq_{\pow{\states}} Q \\
    &\iff \atriple{P}{C}{Q}
\end{align*}
\end{proof}

\todo{Mh not sure it is true}

This means that the validity of Backward Abstract Hoare logic triples on the
opposite powerset of states is equivalent to the same as the one for Abstract 
Hoare logic on the powerset of states, meaning that they prove the same 
properties of programs. In particular, in this context, the two proof systems 
are equivalent since they are both sound and (relatively) complete while 
proving the same properties.

\subsection{Necessary conditions}
Estabished that the dual instantiation of Backward Abstract Hoare logic, we can
look at the backward semantics obtained by not taking the opposite lattice and
by only inverting the semantics of the base commands.

\begin{definition}[Reverse semantics]
  Given an abstract inductive semantics defined on some complete lattice $A$ 
  with base command semantics $\bsem{\cdot}^A$, we can define the reverse
  backward abstract inductive semantics as the backward inductive semantics
  instantiated on the complete lattice $A$ and base command semantics 
  $(\bsem{\cdot}^A)^{-1}$
\end{definition}

Given the usual interpretation on $\pow{\states}$ with semantics of base 
commands $\bsem{\cdot}^{\pow{\states}}$ were $\asem[\pow{\states}]{\cdot}$ is the 
strongest postcondition then the backward abstract inductive semantics on the
the complete lattice $\pow{\states}$ with base command semantics 
\todo{AAAAA} is the backward semantics defined in
\cite{Ascari24}.

\begin{theorem}[Reverse semantics]
  Let $\asemback{\cdot}$ be the reverse semantics of $\asem{\cdot}$ then
  $$\sigma \in \asemback{C}(\{ \sigma' \}) \iff \sigma' \in \asem{C}(\{\sigma\})$$
\end{theorem}
\begin{proof}
  By structural induction on $C$:
  \begin{itemize}
    \item $\sskip$:
      \begin{align*}
        \asemback{\sskip}(\{\sigma'\})
          &= \{\sigma'\}
          &\text{By definition of $\asemback{\cdot}$} \\
          &= \asem{\sskip}(\{\sigma' \})
          &\text{By definition of $\asem{\cdot}$} \\
      \end{align*}
    
    \item $\sskip$:
      \begin{align*}
        \asemback{\sskip}(\{\sigma'\})
          &= \bsem{b}(\{\sigma' \})
      \end{align*}
  \end{itemize}
\end{proof}

Given the usual interpretation on $\pow{\states}$ with semantics of base 
commands $\bsem{\cdot}^{\pow{\states}}$ were $\asem[\pow{\states}]{\cdot}$ is the 
strongest postcondition then the backward abstract inductive semantics on the
the complete lattice $\pow{\states}$ with base command semantics 
$(\bsem{\cdot}^{\pow{\states}})^{-1}$ is the backward semantics defined in
\cite{Ascari24}.
