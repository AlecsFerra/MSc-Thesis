\chapter{Backward Abstract Hoare Logic}

When defining the semantics for $\lang$, we implicitly assumed that the
abstract inductive semantics is forward by giving the definition $\asem{C_1
\fcmp C_2} \defeq \asem{C_2} \circ \asem{C_1}$. However, except for the rule
($\fcmp$), we never explicitly used this fact. We can reuse the theory of
Abstract Hoare logic to construct a slight variation called Backward Abstract
Hoare logic, which describes Hoare logics where the semantics is defined
backward.

\section{Framework}

\subsection{Backward abstract inductive semantics}

To define the backward version of Abstract Hoare logic, we first need a
backward version of the abstract inductive semantics:

\begin{definition}[Backward abstract inductive semantics]
  Given a complete lattice $A$ and a family of monotone functions $\bsem{b}^A :
  A \to A$ for all $b \in Base$, the abstract inductive semantics is defined as
  follows:

  \begin{align*}
      \asemback{\cdot}         & \;\;:\; \lang \to A \to A \\
      \asemback{\sskip}        &\defeq id \\
      \asemback{b}             &\defeq \bsem{b}^A \\
      \asemback{C_1 \fcmp C_2} &\defeq \asemback{C_1} \circ \asemback{C_2} \\
      \asemback{C_1 + C_2}     &\defeq \lambda P . \asemback{C_1} P \join_A \asemback{C_2} P \\
      \asemback{C^\fix}        &\defeq \lambda P . \lfp(\lambda P'. P \join_A \asemback{C} P')
  \end{align*}
\end{definition}

The only difference from the abstract inductive semantics provided in
definition \ref{def:abstract-inductive-semantics} is the case for $C_1 \fcmp
C_2$.

We can prove that the backward abstract inductive semantics is still monotone.

\begin{theorem}[Monotonicity]
  \label{thm:asem-mono-back} 
  For all $C \in \lang$, $\asemback{C}$ is monotone.
\end{theorem}

\begin{proof}
  We modify the inductive case of the proof of theorem \ref{thm:asem-mono} by
  providing only the case for $\asemback{C_1 \fcmp C_2}$ as all the other cases
  are identical.

  \begin{itemize}
    \item $C_1 \fcmp C_2$:

      By inductive hypothesis, $\asemback{C_2}$ is monotone, hence
      $\asemback{C_2}(P) \leq_A \asemback{C_2}(Q)$.

      \begin{align*}
        \asemback{C_1 \fcmp C_2}(P) 
          &= \asemback{C_1}(\asemback{C_2}(P))
          &\text{By definition of $\asemback{C_1 \fcmp C_2}$}\\
          &\leq_A \asemback{C_1}(\asemback{C_2}(Q))
          &\text{By inductive hypothesis on $\asemback{C_1}$} \\
      \end{align*}
  \end{itemize}
\end{proof}

\begin{lemma}[$\asemback{\cdot}$ well-defined]
  For all $C \in \lang$, $\asemback{C}$ is well-defined.
\end{lemma}
\begin{proof}
  From theorems \ref{thm:asem-mono-back} and \ref{thm:knaster}, all the least
  fixpoints in the definition of $\asemback{C^\fix}$ exist. For all the other
  commands, the semantics is trivially well-defined.
\end{proof}

\subsection{Backward Abstract Hoare Logic}

Now we can provide the definition for the backward abstract Hoare triples,
which is the same as for abstract Hoare triples, only with the backward
abstract inductive semantics instead of the usual abstract inductive semantics.

\begin{definition}[Backward Abstract Hoare triple]
  \label{def:baht}
  Given an abstract inductive semantics $\asemback{\cdot}$ on the complete
  lattice $A$, the abstract Hoare triple written $\atripleback{P}{C}{Q}$ is
  valid if and only if $\asemback{C}(P) \leq_A Q$.

  $$\atripleback{P}{C}{Q} \iff \asemback{C}(P) \leq_A Q$$
\end{definition}

Clearly, the proof system only needs to be modified to accommodate the new
semantics for program composition, and all the other rules remain the same.

\begin{definition}[Backward Abstract Hoare rules]$\;$\\
  We only provide the rule for program composition; all the other rules are
  identical to those provided in definition \ref{def:ahtrules}.

  % Rule for sequential composition
  \begin{prooftree}
    \AxiomC{$\vdash \atripleback{P}{C_2}{Q}$}
    \AxiomC{$\vdash \atripleback{Q}{C_1}{R}$}
    \RightLabel{$(\mathbb{\fcmp})$}
    \BinaryInfC{$\vdash \atripleback{P}{C_1 \fcmp C_2}{R}$}
  \end{prooftree}
  If executing $C_2$ from state $P$ leads to state $Q$, and executing $C_1$
  from state $Q$ leads to state $R$, then executing $C_2$ followed by $C_1$ from
  state $P$ leads to state $R$. 
\end{definition}

We can prove again that the proof system is still sound and complete.

\begin{theorem}[Soundness]
  \label{thm:atriple-sound-back}
  $$\vdash \atripleback{P}{C}{Q} \implies \atripleback{P}{C}{Q}$$
\end{theorem}

\begin{proof}
  We modify the inductive case of the proof of theorem \ref{thm:atriple-sound}
  by providing only the case for rule $(\fcmp)$ as all the other cases are
  identical.

  \begin{itemize}
      \item $(\fcmp)$: Then the last step in the derivation was:
        \begin{prooftree}
          \AxiomC{$\vdash \atripleback{P}{C_2}{Q}$}
          \AxiomC{$\vdash \atripleback{Q}{C_1}{R}$}
          \RightLabel{$(\mathbb{\fcmp})$}
          \BinaryInfC{$\vdash \atripleback{P}{C_1 \fcmp C_2}{R}$}
        \end{prooftree}
          
        By inductive hypothesis:
        $\asemback{C_2}(P) \leq_A Q$ and
        $\asemback{C_1}(Q) \leq_A R$.

        The triple is valid since:
        \begin{align*}
          \asemback{C_1 \fcmp C_2}(P)
            &= \asemback{C_1}(\asemback{C_2}(P))
            &\text{By definition of $\asemback{\cdot}$} \\
            &\leq_A \asemback{C_1}(Q)
            &\text{By monotonicity of $\asemback{\cdot}$} \\
            &\leq_A R
        \end{align*}
  \end{itemize}
\end{proof}

\begin{theorem}[Relative $\asemback{\cdot}$-completeness]
  \label{thm:post-completeness-back}
  $$\vdash \atripleback{P}{C}{\asemback{C}(P)}$$
\end{theorem}

\begin{proof}
  We modify the inductive case of the proof of theorem
  \ref{thm:post-completeness} by providing only the case for $C_1 \fcmp C_2$ as
  all the other cases are identical.

  \begin{itemize}
      \item $C_1 \fcmp C_2$:
        By definition $\asemback{C_1 \fcmp C_2}(P) = \asemback{C_1}(\asemback{C_2}(P))$

        \begin{prooftree}
          \AxiomC{(Inductive hypothesis)}
          \noLine
          \UnaryInfC{$\vdash \atripleback{P}{C_2}{\asemback{C_2}(P)}$}
          \AxiomC{(Inductive hypothesis)}
          \noLine
          \UnaryInfC{$\vdash \atripleback{\asemback{C_2}(P)}{C_1}{\asemback{C_1}(\asemback{C_2}(P))}$}
          \RightLabel{($\fcmp$)}
          \BinaryInfC{$\vdash \atripleback{P}{C_1 \fcmp C_2}{\asemback{C_1 \fcmp C_2}(P)}$}
        \end{prooftree}
  \end{itemize}
\end{proof}


\subsection{Backward-Forward Abstract Inductive Semantics Duality}

\begin{definition}[Reverse semantics]
  Given an abstract inductive semantics defined on some complete lattice $A$ 
  with base command semantics $\bsem{\cdot}^A$, we can define the reverse
  backward abstract inductive semantics as the backward inductive semantics
  instantiated on the complete lattice $A$ and base command semantics 
  as the weakest precondition of $\bsem{\cdot}^A$.
\end{definition}

If we interpret the abstract inductive semantics as the strongest postcondition
on the domain $A$, then the backward abstract inductive semantics on the domain 
$A$ is the weakest precondition on the domain $A$.

\begin{observation}
  \label{obs:weakest-precondition}
  Following observation \ref{obs:post}, the reverse backward 
  abstract inductive semantics on the domain $\pow{\states}$ is the weakest 
  precondition.
\end{observation}

But we can do something more if in addition to reversing the semantics of the 
base commands we invert the lattice structure, when executing a non 
deterministic choice instead of taking the join of the two different execution
we take their meet.

\begin{definition}[Dual semantics]
  Given an abstract inductive semantics defined on some lattice $A$ with
  base commands semantics $\bsem{\cdot}^A$, we can define the dual backward 
  abstract inductive semantics as the backward abstract inductive semantics
  instantiated on the lattice $A^{op}$ where the semantics of the
  base commands is defined as $\bsem{\cdot}^{A^{op}}(Q) = wlp(\bsem{\cdot}^A, Q)$.
\end{definition}

If we interpret the abstract inductive semantics as the strongest postcondition
on the domain $A$, then the backward abstract inductive semantics on the domain 
$A$ is the weakest liberal precondition on the domain $A$, this is true since on 
the opposite lattice the role of meets and joins is inverted ($\join_A^{op} = 
\meet_A$ and $\lfp_{A^{op}} = \gfp_A$).

\begin{observation}
  \label{obs:weakest-liberal-precondition}
  Following observation \ref{obs:post}, the dual backward 
  abstract inductive semantics on the domain $\pow{\states}^{op}$ is the weakest 
  liberal precondition.
\end{observation}

For a complete introduction on the differences between the weakest precondition
and the weakest liberal precondition and expecially their definition as
predicate transformes see Chapter 2 of \cite{Kaminski19}.

\section{Instantiations}

\subsection{Necessary conditions}

Using the instantation of the backward abstract inductive semantics of 
observation \ref{obs:weakest-precondition} the reverse abstract inductive semantics
defined on $\pow{\states}$ correspond to the weakest-precondition, when 
instantiating Backward Abstract Hoare logic we obtain a sound and (relative) 
complete logic to reason about the over approximation of the necessary 
conditions to reach a state:

\begin{align*}
  \atripleback[\pow{states}]{Q}{C}{P} 
    & \iff \asemback[\pow{\states}]{C}(Q) \leq P \\
    & \iff wp(C, Q) \subseteq P
\end{align*}

This logic was already studied in \cite{Ascari24} under the name of NC, 
but no proof system was provided, in particular there is proved the connection 
between Hoare and the Necessary Conditions Logics

\begin{theorem}[Connection between Hoare and the Necessary Conditions Logics]
  $$\atriple[\pow{\states}]{P}{C}{Q} \iff \atripleback[\pow{\states}]{\neg Q}{C}
  {\neg P}$$
\end{theorem}

\subsection{Necessary liberal preconditions}

Following what we did for Necessary conditions we can do the same but with the
dual abstract inductive semantics. From observation 
\ref{obs:weakest-liberal-precondition} we know that the dual abstract inductive 
semantics defined on $\pow{\states}^{op}$ correspond to the weakest liberal 
precondition, when instantiating Backward Abstract Hoare logic we obtain a sound
and (relative) complete logic to reason about under-approximation of partial 
necessary conditions.

\begin{align*}
  \atripleback[\pow{states}^{op}]{Q}{C}{P}
    & \iff \asemback[\pow{\states}]{C}(Q) \leq_{\pow{\states}^{op}} P \\
    & \iff \asemback[\pow{\states}]{C}(Q) \geq{\pow{\states}} P \\
    & \iff wlp(C, Q) \supseteq P \\
\end{align*}

This logic was already introduced in \cite{Zhang22} under the name of ???, 
but no proof system was provided, An equivalent theorem as the one for
necessary condition but relating the total correctness (non partial correctness
as Hoare logic, that is already related to necessary conditions) triples to the 
negation (as negation of bot the pre and the post condition) of necessary liberal 
precondition triples.
