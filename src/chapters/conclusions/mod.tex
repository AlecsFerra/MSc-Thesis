\chapter{Conclusions}

We extended traditional Hoare logic by transforming it into a more abstract and
versatile framework. By incorporating principles from abstract interpretation,
we developed a method for reasoning about a broader range of properties.

Through our discussion, we demonstrated that multiple program logics known in
literature are actually special cases of Abstract Hoare Logic. And notably, 
while constructing a program logic for hyperproperties within this framework,
we provided a novel compositional definition of the strongest hyper
postcondition.

Furthermore, we showed how the core proof principles of Hoare logic can be
applied to proving an underapproximation of program properties, highlighting
that the main differences of Incorrectness Logic core proof system, specifically 
suggesting that the infinitary loop rule required for relative completeness:

\begin{prooftree}
  \AxiomC{$[p(n)] \; C \; [p(n + 1)]$}
  \UnaryInfC{$[p(0)] \; C^\star \; [ \exists n . p(n)]$}
\end{prooftree}

Is not due to the logic trying to prove an underapproximation but rather
because it is a total correctness logic, inherently carrying a proof of
termination:


We also discussed the requirements to introduce frame-like rules in Abstract
Hoare Logic and how to obtain a backward variant of this framework. 

\section{Future work}
The following are only preliminary results that we weren't able to explore
because of time constraints.

\subsection{Total correctness/Incorrectness logics}
We have seen how all the partial correctness/incorrectness triples are
instances of (backward) Abstract Hoare Logic and use a very similar proof
system. To complete the picture presented in \cite{Zhang22}, we are missing the
total correctness and incorrectness logics, ??? and `?`?`?. Since they are 
related in the same way as the partial correctness/incorrectness logics are 
related, the same abstraction used to transform Hoare Logic into Abstract Hoare 
Logic could be used to transform Incorrectness Logic \cite{Moller21} into 
Abstract Incorrectness Logic. By abstracting the proof system, we can obtain a 
sound and relatively complete proof system for Abstract Incorrectness Logic. 
Then, by reusing the same technique that we did for Abstract Hoare Logic by 
inverting the semantics and the lattice we can obtain all the four program 
logics that are missing and complete the whole picture.

\subsection{Hyper domains}
We used hyper domains to encode the strongest hyper postcondition and obtained
a Hoare-like logic for hyperproperties. We have shown how we can use the
abstract inductive semantics to model the strongest liberal postcondition,
weakest precondition, and weakest liberal precondition. We could perform the
same trick with the abstract inductive semantics instantiated with a hyper
domain of $\pow{\states}$ and see if it leads to some interesting logics or if
they are all equivalent (since hyperproperties can negate themselves).

Another pain point of the hyper Hoare logic obtained via Abstract Hoare Logic
is that the assertion language is relatively low-level, making it cumbersome to
use for proving actual hyperproperties. The proof system in \cite{Darnier2023}
is actually quite similar to the one obtained with the hyper domains, but the
Exist rule is missing. Since it is necessary to prove the completeness of Hyper
Hoare Logic, it must be embedded somewhere in the rules of Abstract Hoare
Logic.

\subsection{Unifying Forward and Backward Reasoning}
The only difference between Abstract Hoare Logic and Backward Abstract Hoare
Logic lies in the abstract inductive semantics, where the semantics of program
composition ($\fcmp$) is inverted. A potential solution would be to make the
semantics parametric on the composition $\asem{C_1 \fcmp C_2} \defeq \asem{C_1}
\star \asem{C_2}$ and let $\star = \circ^{-1}$ for the forward semantics and
$\star = \circ$ for the backward semantics. However, this approach is somewhat
inelegant when defining the command composition rule for the proof system.

\section{Related work}

The idea of systematically constructing program logics is not new. Kleene
Algebra with Tests (KAT) \cite{Kozen97} was one of the first works of this
kind. In section \ref{chp:join-meet-rules}, we discussed how, in general, we
cannot distribute the non-deterministic choice ($\asem{(C_1 + C_2) \fcmp C_3} \neq
\asem{(C_1 \fcmp C_3) + (C_2 \fcmp C_3)}$), thus violating one of the axioms of
Kleene algebras. Another similar alternative was proposed in \cite{Martin06},
using traced monoidal categories to encode properties of the program. For
example, the monoidal structure is used to model non-deterministic choice but
imposes the same distributivity requirements as Kleene Algebras (this is caused
by $\oplus$ being a bifunctor). However, disregarding expressivity, the main
difference lies in the philosophy behind the approach. Abstract Hoare Logic is
a more semantics-centered approach instead of being an "equational" theory like
KAT. This semantics-centered approach was also vital in providing the idea that
abstract inductive semantics could be used not only to encode the strongest
postcondition but also the strongest liberal postcondition, weakest
precondition, and weakest liberal precondition, thereby unifying all partial
Hoare-like logics.

A more similar approach to that of Abstract Hoare Logic is Outcome Logic
\cite{Zilberstein23}. Like Abstract Hoare Logic, the semantics of the language
in Outcome Logic is parametric on the domain of execution, but the assertion
language is fixed if we ignore the basic assertions on program states. Outcome
Logic originally aimed to unify correctness and incorrectness reasoning with
the powerset instantiation, not to be a minimal theory for sound and complete
Hoare-like logics. In fact, the relative completeness proof is missing. As
discussed in \cite{Darnier2023}, Outcome Logic with the powerset instantiation
is actually a proof system for 2-hyperproperties (hyperproperties regarding at
most two executions). Thus, Outcome triples can be proved in the instantiation
of Abstract Hoare Logic provided in section \ref{chp:hyper}, even though it
would be interesting to find a direct encoding of Outcome Logic in terms of
Abstract Hoare Logic.
