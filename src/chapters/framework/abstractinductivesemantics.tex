\section{Abstract inductive semantics}

From the theory of abstract interpretation we know that the definition of the
denotational semantics can be modified to work on any complete lattice as long
as we provide suitable function for the basic commands. The rationale
behind is the same as in the denotational semantics but instead of representing
collections of states with $\pow{\states}$ now they are represented in an
arbitrary complete lattice.

\begin{definition}[Abstract inductive semantics]

  \label{def:abstract-inductive-semantics}
  Given a complete lattice $A$ and a family of monotone functions $\bsem{\cdot}^A : 
  BCmd \to A \to A$ the abstract inductive semantics is defined inductively as 
  follows:

  \begin{align*}
      \asem{\cdot}         & \;\;:\; \lang \to A \to A \\
      \asem{\sskip}         &\defeq id \\
      \asem{b}             &\defeq \bsem{b}^A \\
      \asem{C_1 \fcmp C_2} &\defeq \asem{C_2} \circ \asem{C_1} \\
      \asem{C_1 + C_2}     &\defeq \lambda P . \asem{C_1} P \join_A \asem{C_2} P \\
      \asem{C^\fix}        &\defeq \lambda P . \lfp(\lambda P'. P \join_A \asem{C} P')
  \end{align*}
\end{definition}


When designing abstract interpreters to perform abstract interpretation,
iterative commands are usually not expressed directly as fixpoints but by
some over-approximation, as is the case for the $C^\fix$ command. This is
necessary since the goal of the abstract interpreter is to be executed and, in
general, if the lattice on which the interpretation executed run has infinite
ascending chains, it's computation can diverge. In our case, the termination
requirement isn't necessary since we are not interested in computing the abstract
inductive semantics but using it as a reference on which the definition of
abstract Hoare logic is dependent.


As we did for the concrete collecting semantics, we need to prove that the
semantics is well-defined. In general, if we drop the requirement for $A$
to be a complete lattice or for $\bsem{b}$ to be monotone, the least fixpoint
could be undefined.

\begin{theorem}[Monotonicity]
  \label{thm:asem-mono} 
  $\forall \; C \in \lang$ $\asem{C}$ is well-defined and monotone.
\end{theorem}
\begin{proof}
  We want to prove that $\forall P, Q \in A$ and $C \in \lang$
  $$P \leq_A Q \implies \asem{C}(P) \leq_A \asem{C}(Q)$$
  By structural induction on $C$:
  \begin{itemize}
    \item $\sskip$:
      \begin{align*}
        \asem{\sskip}(P) 
          &= P 
          & \text{[By definition of $\asem{\sskip}$]}\\
          &\leq Q \\
          &= \asem{\sskip}(Q) 
          & \text{[By definition of $\asem{\sskip}$]}\\
      \end{align*}

    \item $b$:
      \begin{align*}
        \asem{b}(P) 
          &= \bsem{b}^A(P)
          & \text{[By definition of $\asem{b}$]}\\
          &\leq \bsem{b}^A(Q)
          & \text{[By definition]}\\
          &= \asem{b}(Q) 
          & \text{[By definition of $\asem{b}$]}\\
      \end{align*}

    \item $C_1 \fcmp C_2$:

      By inductive hypothesis $\asem{C_1}$ is monotone hence
      $\asem{C_1}(P) \leq_A \asem{C_1}(Q)$

      \begin{align*}
        \asem{C_1 \fcmp C_2}(P) 
          &= \asem{C_2}(\asem{C_1}(P))
          &\text{[By definition of $\asem{C_1 \fcmp C_2}$]}\\
          &\leq_A \asem{C_2}(\asem{C_1}(Q))
          &\text{[By inductive hypothesis on $\asem{C_2}$]} \\
      \end{align*}
  
    \item $C_1 + C_2$:
      \begin{align*}
        \asem{C_1 + C_2}(P) 
          &= \asem{C_1}(P) \join_A \asem{C_2}(P)
          &\text{[By definition of $\asem{C_1 + C_2}$]}\\
          &\leq_A \asem{C_1}(Q) \join_A \asem{C_2}(P)
          &\text{[By inductive hypothesis on $\asem{C_1}$]} \\
          &\leq_A \asem{C_1}(Q) \join_A \asem{C_2}(Q)
          &\text{[By inductive hypothesis on $\asem{C_2}$]} \\
          &= \asem{C_1 + C_2}(Q) 
          &\text{[By definition of $\asem{C_1 + C_2}$]}\\
      \end{align*}
    
    \item $C^\fix$:

      \begin{align*}
        \asem{C^\fix}(P) 
          &= \lfp(\lambda P'. P \join_A \asem{C}(P'))
          &\text{[By definition of $\asem{C^\fix}$]}\\
          &\leq_A \lfp(\lambda P'. Q \join_A \asem{C}(P'))
          &\text{[By theorem \ref{thm:lfp-mono}]}\\
          &= \asem{C^\fix}(Q) 
          &\text{[By definition of $\asem{C^\fix}$]}\\
      \end{align*}
  \end{itemize}


  Clearly all the $\lfp$ are well-defined since by inductive hypothesis
  $\sem{C}$ is monotone and $A$ is a complete from 
  \ref{thm:knaster} the least-fixpoint exists.
\end{proof}

From now on we will refer to the complete lattice $A$ used to define the abstract
inductive semantics as \textit{domain} borrowing the terminology from abstract
interpretation.

\begin{observation}
  \label{obs:post}
  When picking as a domain the lattice $\pow{\states}$ and as basic commands
  $\bsem{b}^{\pow{\states}}(P) = \{ \bsem{b}(\sigma)\downarrow \; \mid \sigma 
  \in P \}$ we will obtain the denotational semantics from the 
  abstract inductive semantics, that is: $\forall \; C \in \lang$ $\forall P \in 
  \pow{\states}$ 
  $$\asem[\pow{\states}]{C}(P) = \sem{C}(P)$$
  This can be easily checked by comparing the two definitions.
\end{observation}

From this observation, we can see that theorem \ref{thm:sem-mono} is just an
instance of theorem \ref{thm:asem-mono} since $\pow{\states}$ is a
complete lattice and the semantics of the basic commands is monotone by
construction.

\subsection{Connection with Abstract Interpretation}

As stated above, the definition of abstract inductive semantics is closely
related to the one of abstract semantics \cite{Cousot77}. In particular, the 
definition of abstract inductive semantics, when the semantics of the basic
commands is sound, is equivalent to an abstract semantics.

\begin{theorem}[Abstract interpretation instance]
  \label{thm:sound-ai}
  If $A$ is an abstract domain and $\bsem{\cdot}^A$ is a sound 
  over-approximation of $\bsem{\cdot}$, then $\asem{\cdot}$ is a sound 
  over-approximation of $\sem{\cdot}$.
\end{theorem}
\begin{proof}
  We prove $\alpha(\sem{C}(P)) \leq \asem{C}(\alpha(P))$ by structural
  induction on $C$:

  \begin{itemize}
    \item $\sskip$:
      \begin{align*}
        \alpha(\sem{\sskip}(P))
          &= \alpha(P)
          & \text{[By definition of $\sem{\sskip}$]}\\
          &= \asem{\sskip}(\alpha(P)) 
          & \text{[By definition of $\asem{\sskip}$]}\\
      \end{align*}

    \item $b$:
      \begin{align*}
        \alpha(\sem{b}(P))
          &= \bsem{b}(P)
          & \text{[By definition of $\sem{b}$]}\\
          &\leq \bsem{b}^A(\alpha(P))
          & \text{[By definition]}\\
          &= \asem{b}(\alpha(P)) 
          & \text{[By definition of $\asem{b}$]}\\
      \end{align*}

    \item $C_1 \fcmp C_2$:

      \begin{align*}
        \alpha(\sem{C_1 \fcmp C_2}(P))
          &= \alpha(\sem{C_2}(\sem{C_1}(P)))
          &\text{[By definition of $\sem{C_1 \fcmp C_2}$]}\\
          &\leq \asem{C_2}(\alpha(\sem{C_1}(P)))
          &\text{[By inductive hypothesis on $C_2$]}\\
          &\leq \asem{C_2}(\asem{C_1}(\alpha(P)))
          &\text{[By inductive hypothesis on $C_1$} \\
          && \text{and $\asem{C_2}$ monotone]}\\
          &= \asem{C_1 \fcmp C_2}(\alpha(P))
          &\text{[By definition of $\asem{C_1 \fcmp C_2}$]}\\
      \end{align*}
  
    \item $C_1 + C_2$:
      \begin{align*}
        \alpha(\sem{C_1 + C_2}(P))
          &= \alpha(\sem{C_1}(P) \cup \sem{C_2}(P))
          &\text{[By definition of $\sem{C_1 + C_2}$]}\\
          &\leq \alpha(\sem{C_1}(P)) \join \alpha(\asem{C_2}(P)) \\
          &\leq \asem{C_1}(\alpha(P)) \join \asem{C_2}(\alpha(P))
          &\text{[By inductive hypothesis on $C_1$} \\
          && \text{and $C_2$]} \\
          &= \asem{C_1 + C_2}(\alpha(P))
          &\text{[By definition of $\asem{C_1 + C_2}$]}\\
      \end{align*}
    
    \item $C^\fix$:

      \begin{align*}
        \alpha(\sem{C^\fix}(P) )
          &= \alpha(\lfp(\lambda P'. P \cup \sem{C}(P')))
          &\text{[By definition of $\sem{C^\fix}$]}\\
          &= \alpha(\bigcup_{n \in \mathbb{N}}{\sem{C}^n(P)}) \\
          &\leq \bigvee_{n \in \mathbb{N}}{\alpha(\sem{C}^n(P))}) \\
          &\leq \bigvee_{n \in \mathbb{N}}{(\asem{C})^n(\alpha(P))}
          &\text{[By inductive hypothesis on $C$]}\\
          &\leq \lfp(\lambda P'. \alpha(P) \join \asem{C}(P')) \\
          &= \asem{C^\fix}(\alpha(P)) 
          &\text{[By definition of $\asem{C^\fix}$]}\\
      \end{align*}
  \end{itemize}
\end{proof}

This connection also allows us to obtain abstract inductive semantics through 
Galois insertions.

\begin{definition}[Abstract Inductive Semantics by Galois Insertion]
  \label{def:aisgi}
  Let $\langle C, \sqsubseteq \rangle \galoiS{\alpha}{\gamma} \langle A, \leq
  \rangle$ be a Galois insertion, and let $\asem[C]{C}$ be some abstract
  inductive semantics defined on $C$. Then, the abstract inductive semantics
  defined on $A$ with $\bsem{b}^A \defeq \alpha \circ \bsem{c}^C \circ \gamma$
  is the abstract inductive semantics obtained by the Galois insertion between
  $C$ and $A$.
\end{definition}

The abstract inductive semantics obtained by Galois insertion between 
$\pow{\states}$ and any domain $A$ corresponds to the best abstract inductive 
interpreter on $A$.

\begin{observation}
  \label{obs:abstract-fix}
  There are some domains where $\exists \; C \in \lang$ such that
  $\bigvee_{n \in \nat} (\asem{C})^n(P) \neq \lfp(\lambda P'. P \join_A
  \asem{C}(P'))$.
\end{observation}
\begin{example}
  Let $C \defeq (x > 1? \fcmp ((even(x) ? \fcmp X := x + 3) +
  (\neg even(x)? \fcmp x := x - 2))^\fix$ when performing the computation on
  the interval domain, if we compute $C$ using the infinitary join:
  \begin{align*}
    \asem{C}([5, 5])
      &= \bigvee_{n \in \mathbb{N}} (\asem{x > 1? \fcmp ((even(x) ? \fcmp x :=
        x + 3) + (\neg even(x)? \fcmp x := x - 2))})^n([5, 5]) \\
      &= [5, 5] \join [3, 3] \join [1, 1] \join \bot \join \bot ... \\
      &= [1, 5]
  \end{align*}

  The difference is caused by the fact that when we are computing the infinite 
  join, all the joins happen after executing the semantics of the loop body.
  
\end{example}
