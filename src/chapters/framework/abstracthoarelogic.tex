\section{Abstract Hoare Logic}
\label{chp:intro-ahorare}

\subsection{Hoare logic}
Hoare logic  \cite{Hoare69, Floyd93} was one of the first method designed for 
the verification of programs, It's core concept is that of partial correctness 
assertions. A Hoare triple is a formula $\htriple{P}{C}{Q}$ where $P$ and $Q$ are 
assertions on the initial and final states of a program $C$, respectively. 
These assertions are partial in the sense that $Q$ is meaningful only when the 
execution of $C$ on $P$ terminates.

Hoare logic is designed as a proof system, where the syntax 
$\vdash \htriple{P}{C}{Q}$ indicates that the triple 
$\htriple{P}{C}{Q}$ is proved by applying the rules of the proof system.

The original formulation of Hoare logic was given for an imperative language 
with deterministic constructs, but it can be easily defined for our language 
$\lang$ following the work in \cite{Moller21}.

\begin{definition}[Hoare triple]
  \label{def:hoare}
  Fixed the semantics of the basic commands, an Hoare triple denoted by 
  $\htriple{P}{C}{Q}$, is valid if and only if $\sem{C}(P) \subseteq Q$.

  $$\models \htriple{P}{C}{Q} \iff \sem{C}(P) \subseteq Q$$
\end{definition}

We will use the syntax $\models \htriple{P}{C}{Q}$ to refer to valid triples,
$\not \models \htriple{P}{C}{Q}$ to refer to invalid triples, and
$\htriple{P}{C}{Q}$ when we are not asserting the validity or invalidity of a
triple.

\begin{example}[Hoare triples]
  \label{exmp:hlogic}
We have that $\htriple{x \in [1, 2]}{x := x + 1}{x \in [2, 3]}$, is a
valid triple since from any state in which either $x = 1$ or $x = 2$,
incrementing by one the value of $x$ leads to states in which $x$ is either $2$
or $3$. Specifically, starting from $x = 1$ leads us to $x = 2$ and starting
from $x = 2$ leads us to $x = 3$.

Since the conclusion of Hoare triples must contain all the final states, the
triple $\htriple{P}{C}{\top}$ is always valid since $\top$ contains all the
possible states.

An example of an invalid triple is $\htriple{x \in [1, 2]}{x := x
+ 1}{x \in [1, 2]}$ since the state $x = 2$ satisfies the precondition and
executing the program on it results in the state $x = 3$, which does not satisfy
$x \in [1, 2]$.

Since Hoare logic is concerned only with termination, when the program is
non-terminating, we can prove any property. For example, $\htriple{x
\in [0, 10]}{(x \leq 20? \fcmp x := x - 1)^\fix \fcmp x \geq 20?}{Q}$ is always
a valid triple since the program is non-terminating for any $x \in [0, 10]$. 
The set of reachable states is empty, thus the postcondition is vacuously true.

This is the reason why Hoare logic is called a partial correctness logic,
where partial means that it can prove the adherence of a program to
some specification only when it is terminating. The termination of the program
must be proved by resorting to some alternative method.
\end{example}

\begin{definition}[Hoare logic]$\;$ \\
  \label{def:hoaretules}
  The rules of Hoare logic are defined as follows:
  \begin{prooftree}
    \AxiomC{$ $}
    \RightLabel{$(\sskip)$}
    \UnaryInfC{$\vdash \htriple{P}{\sskip}{P}$}
  \end{prooftree}

  % Rule for a basic command
  \begin{prooftree}
    \AxiomC{$ $}
    \RightLabel{$(base)$}
    \UnaryInfC{$\vdash \htriple{P}{b}{\bsem{b}(P)}$}
  \end{prooftree}

  % Rule for sequential composition
  \begin{prooftree}
    \AxiomC{$\vdash \htriple{P}{C_1}{Q}$}
    \AxiomC{$\vdash \htriple{Q}{C_2}{R}$}
    \RightLabel{$(seq)$}
    \BinaryInfC{$\vdash \htriple{P}{C_1 \fcmp C_2}{R}$}
  \end{prooftree}

  % Rule for nondeterministic choice
  \begin{prooftree}
    \AxiomC{$\vdash \htriple{P}{C_1}{Q}$}
    \AxiomC{$\vdash \htriple{P}{C_2}{Q}$}
    \RightLabel{$(disj)$}
    \BinaryInfC{$\vdash \htriple{P}{C_1 + C_2}{Q}$}
  \end{prooftree}

  \begin{prooftree}
    \AxiomC{$\vdash \htriple{P}{C}{P}$}
    \RightLabel{$(iterate)$}
    \UnaryInfC{$\vdash \htriple{P}{C^\fix}{P}$}
  \end{prooftree}

  % Rule for strengthening the precondition and weakening the postcondition
  \begin{prooftree}
    \AxiomC{$P \subseteq P'$}
    \AxiomC{$\vdash \htriple{P'}{C}{Q'}$}
    \AxiomC{$Q' \subseteq Q$}
    \RightLabel{$(consequence)$}
    \TrinaryInfC{$\vdash \htriple{P}{C}{Q}$}
  \end{prooftree}
\end{definition}

The proof system described in Definition \ref{def:hoaretules} is logically
sound, meaning that all it's provable triples are valid with respect to
definition \ref{def:hoare}.

\begin{theorem}[Soundness]
  $$\vdash \htriple{P}{C}{Q} \implies \models \htriple{P}{C}{Q}$$
\end{theorem}

As observed by Cook \cite{Cook78}, the reverse implication is not true, in 
general, as a consequence of Gödel's incompleteness theorem. For this reason, 
Cook developed the concept of relative completeness, in which all the instances 
of $\subseteq$ are provided by an oracle, proving that the incompleteness of the 
proof system is only caused by the incompleteness of the assertion language.

\begin{theorem}[Relative completeness]
  \label{thm:hlogic-complete}
  $$\models \htriple{P}{C}{Q} \implies \vdash \htriple{P}{C}{Q}$$
\end{theorem}

\subsection{Abstracting Hoare logic}
The idea of designing a Hoare-like logic to reason about properties of programs
expressible within the theory of lattices using concepts from abstract
interpretation is not new. In fact, \cite{Cousot12} already proposed a
framework to perform this kind of reasoning. However, the validity of the
triples in \cite{Cousot12} dependends on the standard definition of Hoare
triples, and the proof system is incomplete if we ignore the rule to embed
standard Hoare triples in the abstract ones.

Our approach will be different. In particular, the meaning of abstract Hoare 
triples will be dependent on the abstract inductive semantics, and we will 
provide a sound and (relatively) complete without resorting to embedd Hoare 
logic in it's proof system as \cite{Cousot12}.

\begin{definition}[Abstract Hoare triple]
  \label{def:aht}
  Given an abstract inductive semantics $\asem{\cdot}$ on the complete lattice
  $A$, the abstract Hoare triple written $\atriple{P}{C}{Q}$ is valid if
  and only if $\asem{C}(P) \leq_A Q$.

  $$\models \atriple{P}{C}{Q} \iff \asem{C}(P) \leq_A Q$$
\end{definition}

The definition is equivalent to the definition \ref{def:hoare} 
but here the abstract inductive semantics is used to provide the strongest 
postcondition of programs.

In Abstract Hoare logic some of the examples shown in example \ref{exmp:hlogic} 
still hold, in particular we have that:
\begin{example}
  $$\models \atriple{P}{C}{\top}$$
\end{example}
\begin{proof}
  \begin{align*}
    \models \atriple{P}{C}{\top}
      & \iff \asem{C}(P) \leq \top & \text{By definition of $\atriple{\cdot}{\cdot}{\cdot}$} \\
  \end{align*}

  And since $\top$ is the top element of $A$ $\top \geq \asem{C}(P)$
\end{proof}

\subsubsection{2.3.3 Proof system}
As per Hoare logic we will provide a sound and relatively complete (in the sense
of \cite{Cook78}) proof system to derive abstract Hoare triples in a 
compositional fashion.

\begin{definition}[Abstract Hoare rules]$\;$\\
  \label{def:ahtrules}
  % Rule for the identity command
  \begin{prooftree}
    \AxiomC{$ $}
    \RightLabel{$(\sskip)$}
    \UnaryInfC{$\vdash \atriple{P}{\sskip}{P}$}
  \end{prooftree}

  % Rule for a basic command
  \begin{prooftree}
    \AxiomC{$ $}
    \RightLabel{$(b)$}
    \UnaryInfC{$\vdash \atriple{P}{b}{\bsem{b}^A(P)}$}
  \end{prooftree}

  % Rule for sequential composition
  \begin{prooftree}
    \AxiomC{$\vdash \atriple{P}{C_1}{Q}$}
    \AxiomC{$\vdash \atriple{Q}{C_2}{R}$}
    \RightLabel{$(\mathbb{\fcmp})$}
    \BinaryInfC{$\vdash \atriple{P}{C_1 \fcmp C_2}{R}$}
  \end{prooftree}

  % Rule for nondeterministic choice
  \begin{prooftree}
    \AxiomC{$\vdash \atriple{P}{C_1}{Q}$}
    \AxiomC{$\vdash \atriple{P}{C_2}{Q}$}
    \RightLabel{$(+)$}
    \BinaryInfC{$\vdash \atriple{P}{C_1 + C_2}{Q}$}
  \end{prooftree}

  % Rule for iteration (Kleene star)
  \begin{prooftree}
    \AxiomC{$\vdash \atriple{P}{C}{P}$}
    \RightLabel{$(\fix)$}
    \UnaryInfC{$\vdash \atriple{P}{C^\fix}{P}$}
  \end{prooftree}

  % Rule for strengthening the precondition and weakening the postcondition
  \begin{prooftree}
    \AxiomC{$P \leq P'$}
    \AxiomC{$\vdash \atriple{P'}{C}{Q'}$}
    \AxiomC{$Q' \leq Q$}
    \RightLabel{$(\leq)$}
    \TrinaryInfC{$\vdash \atriple{P}{C}{Q}$}
  \end{prooftree}
\end{definition}

The rules can be summarized as:
\begin{itemize}
  \item The identity command does not change the state, so if $P$ holds before,
    it will hold after the execution.

  \item For a basic command $b$, if $P$ holds before the execution, then 
    $\bsem{b}^A(P)$ holds after the execution.

  \item If executing $C_1$ from state $P$ leads to state $Q$, and executing
    $C_2$ from state $Q$ leads to state $R$, then executing $C_1$ followed by
    $C_2$ from state $P$ leads to state $R$.

  \item If executing either $C_1$ or $C_2$ from state $P$ leads to state $Q$,
    then executing the nondeterministic choice $C_1 + C_2$ from state $P$ also
    leads to state $Q$.

  \item If executing command $C$ from state $P$ leads back to state $P$, then
    executing $C$ repeatedly (zero or more times) from state $P$ also leads
    back to state $P$.

  \item If $P$ is stronger than $P'$ and $Q'$ is stronger than $Q$, then we can
    derive $\atriple{P}{C}{Q}$ from $\atriple{P'}{C}{Q'}$.
\end{itemize}

Pisnelllo $\gamma + \eta$

The proofsystem is nonother than definition \ref{def:hoaretules},
where the assertion are replaced by elements of the complete lattice $A$.

Note that we denote Abstract Hoare Triples as defined in defintion \ref{def:aht}
with the notation $\atriple{P}{C}{Q}$ while we denote the triples obtained
with the inference rules of definition \ref{def:ahtrules} by $\vdash 
\atriple{P}{C}{Q}$.

The proofsystem for Abstract Hoare logic is sound, as the original Hoare logic.

\begin{theorem}[Soundness]
  \label{thm:atriple-sound}
  $$\vdash \atriple{P}{C}{Q} \implies \models \atriple{P}{C}{Q}$$
\end{theorem}
\begin{proof}
  By structural induction on the last rule applied in the derivation of
  $\vdash \atriple{P}{C}{Q}$:
  \begin{itemize}

    \item $(\sskip)$:
      Then the last step in the derivation was: 
      \begin{prooftree}
        \AxiomC{$ $}
        \RightLabel{$(\sskip)$}
        \UnaryInfC{$\vdash 
          \atriple{P}{\sskip}{P}$}
      \end{prooftree}

      The triple is valid since:
      \begin{align*}
        \asem{\sskip}(P)
          &= P 
          &\text{[By definition of $\asem{\cdot}$]}
      \end{align*}

      \item $(b)$:
        Then the last step in the derivation was:
        \begin{prooftree}
          \AxiomC{$ $}
          \RightLabel{$(b)$}
          \UnaryInfC{$\vdash 
          \atriple{P}{b}{\bsem{b}^A(P)}$}
        \end{prooftree}

        The triple is valid since:
        \begin{align*}
          \asem{b}(P)
            &= \bsem{b}^A(P)
            & \text{[By definition of $\asem{\cdot}$]}
        \end{align*}

      \item $(\fcmp)$: Then the last step in the derivation was:
        \begin{prooftree}
          \AxiomC{$\vdash \atriple{P}{C_1}{Q}$}
          \AxiomC{$\vdash \atriple{Q}{C_2}{R}$}
          \RightLabel{$(\mathbb{\fcmp})$}
          \BinaryInfC{$\vdash \atriple{P}{C_1 \fcmp C_2}
            {R}$}
        \end{prooftree}
          
        By inductive hypothesis:
        $\asem{C_1}(P) \leq_A Q$ and
        $\asem{C_2}(Q) \leq_A R$.

        The triple is valid since:
        \begin{align*}
          \asem{C_1 \fcmp C_2}(P)
            &= \asem{C_2}(\asem{C_1}(P))
            &\text{[By definition of $\asem{\cdot}$]} \\
            &\leq_A \asem{C_2}(Q)
            &\text{[By monotonicity of $\asem{\cdot}$]} \\
            &\leq_A R
        \end{align*}

      \item $(+)$: Then the last step in the derivation was:
        \begin{prooftree}
          \AxiomC{$\vdash \atriple{P}{C_1}{Q}$}
          \AxiomC{$\vdash \atriple{P}{C_2}{Q}$}
          \RightLabel{$(+)$}
          \BinaryInfC{$\vdash \atriple{P}{C_1 + C_2}{Q}$}
        \end{prooftree}

        By inductive hypothesis: $\asem{C_1}(P) \leq Q$ and
        $\asem{C_2}(P) \leq Q$.

        The triple is valid since:
        \begin{align*}
          \asem{C_1 + C_2}(P)
            &= \asem{C_1}(P) \join_A \asem{C_2}(P)
            &\text{[By definition of $\asem{\cdot}$]} \\
            &\leq_A Q \join_A Q \\
            &= Q
        \end{align*}

      \item $(\fix)$:
        Then the last step in the derivation was:
        \begin{prooftree}
          \AxiomC{$\vdash \atriple{P}{C}{P}$}
          \RightLabel{$(\fix)$}
          \UnaryInfC{$\vdash \atriple{P}{C ^ \lfp}{P}$}
        \end{prooftree}

        By inductive hypothesis: $\asem{C}P \leq P$

        \begin{align*}
          \asem{C^\fix}(P)
            &= \lfp(\lambda P' \to P \join_A \asem{C}(P')) \\
        \end{align*}

        We will show that $P$ is a fixpoint of 
        $\lambda P' \to P \join_A \asem{C}(P')$.

        \begin{align*}
          (\lambda P' \to P \join_A \asem{C}(P'))(P)
            &= P \join_A \asem{C}(P)
            & \text{[since $\asem{C}(P) \leq P$]} \\
            &= P
        \end{align*}

        Hence $P$ is a fixpoint of $\lambda P' \to P \join_A \asem{C}(P')$,
        therefore it's at leas as big as the least one, $\lfp(\lambda P' \to P 
        \join_A \asem{C}(P')) \leq_A P$ thus making the triple valid.

      \item $(\leq)$: Then the last step in the derivation was:
        \begin{prooftree}
          \AxiomC{$P \leq P'$}
          \AxiomC{$\vdash \atriple{P'}{C}{Q'}$}
          \AxiomC{$Q' \leq Q$}
          \RightLabel{$(\leq)$}
          \TrinaryInfC{$\vdash \atriple{P}{C}{Q}$}
        \end{prooftree}

        By inductive hypothesis: $\asem{C}(P') \leq_A Q'$.
        
        The triple is valid since:
        \begin{align*}
          \asem{C}(P)
            & \asem{C}(P')
            & \text{[By monotonicity of $\asem{\cdot}$]}\\
            & \leq_A Q' 
            & \text{[By inductive hypothesis]} \\
            & \leq_A Q
        \end{align*}
  \end{itemize}
\end{proof}

The proof system turns out to be relatively complete aswell, in the sense that
the axioms are complete relative to what we can prove in the underlying
assertion language, that in our case is described by the complete lattice.

We will first prove a slightly weaker result, where we will show that we can
prove the strongest post-condition of every program.

\begin{theorem}[Relative $\asem{\cdot}$-completeness]
  \label{thm:post-completeness}
  $$\vdash \atriple{P}{C}{\asem{C}(P)}$$
\end{theorem}
\begin{proof}
  By structural induction on $C$:
  \begin{itemize}

    \item $\sskip$:
      By definition $\asem{\sskip}(P) = P$
      \begin{prooftree}
        \AxiomC{$ $}
        \RightLabel{$(\sskip)$}
        \UnaryInfC{$\vdash \atriple{P}{\sskip}{P}$}
      \end{prooftree}

      \item $b$:
        By definition $\asem{b}(P) = \bsem{b}^A(P)$
        \begin{prooftree}
          \AxiomC{$ $}
          \RightLabel{$(b)$}
          \UnaryInfC{$\vdash \atriple{P}{b}{\bsem{b}^A(P)}$}
        \end{prooftree}

      \item $C_1 \fcmp C_2$:
        By definition $\asem{C_1 \fcmp C_2}(P) = 
        \asem{C_2}(\asem{C_1}(P))$

        \begin{prooftree}
          \AxiomC{(Inductive hypothesis)}
          \noLine
          \UnaryInfC{$\vdash \atriple{P}{C_1}{\asem{C_1}(P)}$}
          \AxiomC{(Inductive hypothesis)}
          \noLine
          \UnaryInfC{$\vdash \atriple{\asem{C_1}(P)}{C_2}
            {\asem{C_2}(\asem{C_1}(P))}$}
          \RightLabel{$(\fcmp)$}
          \BinaryInfC{$\vdash \atriple{P}{C_1 \fcmp C_2}
            {\asem{C_2}(\asem{C_1}(P))}$}
        \end{prooftree}


      \item $C_1 + C_2$:
        By definition $\bsem{C_1 + C_2}(P) = 
        \bsem{C_1}(P) \join_A \bsem{C_2}(P)$

        \begin{prooftree}
          \AxiomC{$\pi_1$}
          \AxiomC{$\pi_2$}
          \RightLabel{$(+)$}
          \BinaryInfC{$\vdash \atriple{P}{C_1 + C_2}
            {\asem{C_1}(P) \join_A \asem{C_2}(P)}$}
        \end{prooftree}

        Where $\pi_1$:
        \begin{prooftree}
          \AxiomC{$P \leq_A P$}
          \AxiomC{(Inductive hypothesis)}
          \noLine
          \UnaryInfC{$\vdash \atriple{P}{C_1}{\asem{C_1}(P)}$}
          \AxiomC{$\asem{C_1}(P) \leq_A \asem{C_1}(P) \join_A \asem{C_2}(P)$}
          \RightLabel{$(\leq)$}
          \TrinaryInfC{$\vdash \atriple{P}{C_1}
            {\asem{C_1}(P) \join_A \asem{C_2}(P)}$}
        \end{prooftree}

        and $\pi_2$:
        \begin{prooftree}
          \AxiomC{$P \leq_A P$}
          \AxiomC{(Inductive hypothesis)}
          \noLine
          \UnaryInfC{$\vdash \atriple{P}{C_2}{\asem{C_2}(P)}$}
          \AxiomC{$\asem{C_2}(P) \leq_A \asem{C_1}(P) \join_A \asem{C_2}(P)$}
          \RightLabel{$(\leq)$}
          \TrinaryInfC{$\vdash \atriple{P}{C_2}
            {\asem{C_1}(P) \join_A \asem{C_2}(P)}$}
        \end{prooftree}

      \item $C^\fix$:
        By definition $\bsem{C^\fix}(P) = lfp(\lambda P' \to P \join_A
        \asem{C}(S')$.

        Let $K \defeq lfp(\lambda P' \to P \join_A \asem{C}(S')$
        hence $K = P \join_A \asem{C}(K)$ since it is a fixpoint, thus
        \begin{itemize}
          \item $\alpha_1$: $K \geq_A P$
          \item $\alpha_2$: $K \geq_A \asem{C}(K)$
        \end{itemize}

          \begin{prooftree}
            \AxiomC{$\alpha_1$}
            \AxiomC{$K \leq_A K$}
            \AxiomC{(Inductive hypothesis)}
            \noLine
            \UnaryInfC{$\vdash \atriple{K}{C}{\asem{C}(K)}$}
            \AxiomC{$\alpha_2$}
            \TrinaryInfC{$\vdash \atriple{K}{C}{K}$}
            \RightLabel{$(\fix)$}
            \UnaryInfC{$\vdash \atriple{K}{C^\fix}{K}$}
            \AxiomC{$K \leq_A K$}
            \RightLabel{$(\leq)$}
            \TrinaryInfC{$\vdash \atriple{P}{C^\fix}{K}$}
          \end{prooftree}
  \end{itemize}
\end{proof}

We can now show the relative completeness, by applying the rule $(\leq)$ to
achieve the desired post-condition.

\begin{theorem}[Relative completeness]
  \label{thm:completeness}
  $$\models \atriple{P}{C}{Q} \implies \vdash \atriple{P}{C}{Q}$$
\end{theorem}
\begin{proof}
  By definition of $\models \atriple{P}{C}{Q} \iff Q \geq_A \asem{C}(P)$

  \begin{prooftree}
    \AxiomC{$P \leq_A P$}
    \AxiomC{(By Theorem \ref{thm:post-completeness})}
    \noLine
    \UnaryInfC{$\vdash \atriple{P}{C}{\asem{C}(P)}$}
    \AxiomC{$Q \geq_A \asem{C}(P)$}
    \RightLabel{$(\leq)$}
    \TrinaryInfC{$\vdash \atriple{P}{C}{Q}$}
  \end{prooftree}
\end{proof}
