\chapter{Instantiating Abstract Hoare Logic}

In this chapter, we will demonstrate how to instantiate abstract Hoare logic to
create new proof systems. We will also illustrate that the framework of
abstract Hoare logic is sufficiently general to reason about properties that
cannot be expressed in standard Hoare logic.

\subimport{./}{hoare}
\subimport{./}{interval}
\subimport{./}{hyper_hoare}

\section{Partial Incorrectness}
\label{chp:partial-incorrectness}

Any instantiation of the abstract inductive semantics provides us with another
instantiation for free, as the semantics is parameterized by a complete lattice
\( A \), and the dual of a complete lattice \( A \) is also complete. Therefore,
we can derive the dual abstract inductive semantics on the complete lattice
\( A^{op} \).

\begin{definition}[Dual Abstract Inductive Semantics]
  Given an abstract inductive semantics defined on a complete lattice \( A \)
  with basic commands \( \bsem{\cdot}^A \), the dual abstract inductive
  semantics is defined on the complete lattice \( A^{op} \) with basic command
  semantics \( \bsem{\cdot}^{A^{op}} = \bsem{\cdot}^A \).
\end{definition}

Since the dual abstract inductive semantics is itself an abstract inductive
semantics, it naturally induces an Abstract Hoare Logic. In the dual lattice,
where the partial order is inverted, operations such as joins and meets are
reversed, leading to an inversion of \( \lfp \) and \( \gfp \). Hence, the dual
abstract inductive semantics, viewed from the dual lattice, can be formulated
as follows:
\begin{align*}
  \asem[A^{op}]{\sskip}        &= id &&= id \\
  \asem[A^{op}]{b}             &= \bsem{b}^{A^{op}} &&= \bsem{b}^A \\
  \asem[A^{op}]{C_1 \fcmp C_2} &= \asem[A^{op}]{C_2} \circ \asem[A^{op}]{C_1} &&= \asem[A^{op}]{C_2} \circ \asem[A^{op}]{C_1} \\
  \asem[A^{op}]{C_1 + C_2}     &= \lambda P . \asem[A^{op}]{C_1} P \join_{A^{op}} \asem[A^{op}]{C_2} P &&= \lambda P . \asem[A^{op}]{C_1} P \meet_A \asem[A^{op}]{C_2} P \\
  \asem[A^{op}]{C^\fix}        &= \lambda P . \lfp_{A^{op}}(\lambda P'. P \join_{A^{op}} \asem[A^{op}]{C} P') &&= \lambda P . \gfp_{A}(\lambda P'. P \meet_A \asem[A^{op}]{C} P')
\end{align*}

Interpreting the dual abstract inductive semantics, we understand that in the
dual lattice \( A^{op} \), non-deterministic choices are handled by taking the
meet of two branches, reflecting certainty rather than possibility. Instead of
considering all reachable states (union of states reached by each branch), it
considers the intersection of states guaranteed to be reached by both branches.
This inversion similarly applies to the \( \fix \) command.

Given the inverted order in the dual lattice, the validity of Abstract Hoare
triples is reversed:
$$\models \atriple[A^{op}]{P}{C}{Q} \iff \asem{C}(P) \leq_{A^{op}} Q \iff \asem[A^{op}]{C}(P) \geq_A Q$$

When deriving the dual abstract inductive semantics from the abstract
inductive semantics on \( \pow{\states} \) (strongest postcondition), the dual
semantics corresponds to the strongest liberal postcondition as introduced in
\cite{Zhang22} (in the boolean case). These triples are termed "partial
incorrectness," implying that if \( \models \atriple[A^{op}]{Q}{C}{P} \), then
\( P \) over-approximates the states reaching \( Q \), accounting for
termination. This concept aligns with "Necessary Preconditions" explored in
\cite{Cousot13}, and Abstract Hoare Logic provides a sound and complete
proof system for this logic.
